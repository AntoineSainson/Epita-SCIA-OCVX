\documentclass[11pt, a4paper]{article}

\usepackage[french]{babel}
\usepackage{fancyhdr}
\usepackage[margin=.8in]{geometry}

\usepackage{Style/TeXingStyle}

\pagestyle{fancy}
\renewcommand{\headrulewidth}{1.5pt}
\renewcommand{\footrulewidth}{0.5pt}
\fancyhead[L]{EPITA\_ING2\_2019\_S8}
\fancyhead[R]{Majeure SCIA}
\fancyhead[C]{OCVX1}
\fancyfoot[C]{\thepage}
\fancyfoot[L]{2018}
\fancyfoot[R]{\textbf{Chargé de cours :} \textsc{Bashar~DUDIN}}

\pretitle{\vspace{-2.5\baselineskip} \begin{center}}
\title{%
  { \huge Transformations Affines}%
}
\posttitle{
\end{center}
\rule{\textwidth}{1.5pt}
\vspace{-3\baselineskip}
}
\author{}
\date{}

\pdfinfo{
   /Author (Bashar Dudin)
   /Title  (Exercices Geometrie - 2018)
   /Subject (Geometrie Affine)
}

\begin{document}

\maketitle\thispagestyle{fancy}

Cette feuille est centrée autour des applications affines de $\R^n$,
plus particulièrement autours des projections et transformations de
l'espace affine euclidien. L'objectif est de vous apporter les
éléments de langage et les résultats de structure qui vous permettent
de les expliciter et les utiliser dans les contextes de ML. Pour
rappel la majeure partie des algorithmes de ML sont de nature
géométrique.

Nous allons en passant en profiter pour faire quelques rappels de
calcul matriiel ainsi que d'algèbre linéaire quand nécessaire.

\section{Pivot de Gauss}
\label{sec:pivotGauss}

On se donne une matrice $M \in \mc{M}_{m, n}(\R)$. Une opération
élémentaire sur $M$ est une des deux opérations suivantes:
\begin{itemize}
\item[\textbullet]
  pour $i \in \{1, \ldots, m\}$ et $j \in \{1, \ldots, n\}$
  interchanger les lignes $L_i$ et $L_j$
  \[
  L_i \leftrightarrow L_j
  \]
\item[\textbullet]
  pour $i \in \{1, \ldots, m\}$, $j \in \{1, \ldots, n\}$, $i \neq j$,
  $\lambda_i \in \R^*$ et $\lambda_i \in \R$, remplacer la ligne $L_i$
  par $\lambda_i L_i + \lambda_jL_j$ :
  \[
  L_i \leftarrow \lambda_iL_i + \lambda_jL_j.
  \]
\end{itemize}
Le pivot de Gauss est un algorithme de transformation d'une matrice
$M$ en matrice triangulaire supérieur et éventuellement en matrice
identité, dans le but d'inverser une matrice, de calculer un
déterminent ou de résoudre un système linéaire. Il est d'utilisation
constante en calcul numérique.\footnote{L'expression de l'inverse
  d'une matrice inversible à l'aide des cofacteurs est à prohiber!
  Elle est théoriquement intéressante et particulièrement esthétique
  mais numériquement inéfficace.}

\subsection{Opérations élémentaires}
\label{subsec:opElementaires}

On traduit par la suite les opérations élémentaires précédentes à
l'aide du produit matriciel.
\begin{question}
  Trouver des matrices $P(i, j)$ et $U(i, j, \lambda, \nu)$ dont la
  multiplications avec $M$ réalise les opérations élémentaires
  précédentes.
\end{question}

\begin{question}
  Trouver les matrices $P_t(i, j)$ et $U_t(i, j, \lambda, \nu)$ qui
  réalisent les opérations précédentes sur les colonnes.
\end{question}

\subsection{Systèmes linéaires}

\begin{question}
  Déduire de la section \ref{subsec:opElementaires} que les opérations
  élémentaires du pivot de Gausse transforme tout système linéaire
  dont $M$ est une matrice (si appliqué des deux côtés de l'égalité)
  en un système linéaire ayant les mêmes solutions.
\end{question}

\begin{question}
  Comment procéder pour décrire toute les solutions d'un système
  linéaire?
\end{question}

\begin{question}
  Discuter de l'efficacité de l'implémentation de l'algorithme
\end{question}

\subsection{Déterminent}

\begin{question}
  Retrouver les relations standards sur le déterminent d'une matrice
  qui subit une opération élémentaire à l'aide de la formule
  \[
  \det(AB) = \det(A)\det(B).
  \]
\end{question}

\begin{question}
  Comment procéder pour calculer le déterminent d'une matrice à l'aide
  du pivot de Gauss?
\end{question}

\subsection{Matrices équivalentes}

\noindent Deux matrices $A$ et $B$ sont dites équivalentes s'il
existe deux matrices inversibles $P$ et $Q$ telle que
\[
A = PBQ.
\]
\begin{question}
  Justifier le fait que toute matrice est équivalente à une matrice
  ayant un bloc identité et des zéros partout ailleurs.
\end{question}


\section{Applications Affines}
\label{sec:defnappaffine}

Cette section est un kit de survie en milieu hostile. Il s'agit de
s'armer des différentes notions de transformations et applications
affines dont vous aurez l'usage lors de vos cours de ML. Notre
démarche dans la suite est particulièrement pragmatique\footnote{Elle
  pourrait heurter certaines âmes sensibles et quelques matheux qu'on
  garder un peu de sens esthétique.}; on se limite au cas des
applications affines de $\R^n$ dans $\R^m$.
\begin{defn}
  Une application $f : \R^n \to \R^m$ est une application affine s'il
  existe
  \begin{itemize}
  \item[\textbullet]
    des bases $\bs{v} = (v_i)_{i=1}^n$ et
    $\bs{w} = (w_j)_{j=1}^m$ respectivement de $\R^n$ et
    $\R^m$
  \item[\textbullet]
    une matrice $M$ de $\mc{M}_{m, n}(\R)$
  \item[\textbullet]
    un vecteur $b_{\bs{w}}$ écrit en coordonnées dans la base $\bs{w}$
  \end{itemize}
  tels que pour tout vecteur $x_{\bs{v}}$ écrit dans la base $\bs{v}$,
  $f$ prend la forme
  \begin{equation}
    \label{defn:affine}
    f(x_{\bs{v}}) = Mx_{\bs{v}} + b_{\bs{w}}.
  \end{equation}
  L'équation \ref{defn:affine} est dite équation de $f$ dans les bases
  $\bs{v}$ et $\bs{w}$.
\end{defn}
\begin{exmp}
  Une application affine peut être donné par une experssion dans la
  base canonique
  \[
  x \mapsto Mx + b
  \]
  où $x$ et $b$ sont des vecteurs respectivement dans $\R^n$ et $\R^m$
  et $M \in \mc{M}_{m, n}(\R)$. Par exemple
  \[
  \begin{pmatrix} x \\ y \end{pmatrix} \mapsto \begin{pmatrix} 1 & 1
    \\ 2 & -1 \end{pmatrix}\begin{pmatrix} x \\ y \end{pmatrix}
  + \begin{pmatrix} 3 \\ -2 \end{pmatrix}
  \]
  Ce qui nous donne une fonction de $\R^2$ dans $\R^2$ affine en $x$
  et $y$ au sens qu'on entend depuis tout petit.
\end{exmp}
Afin de pouvoir donner des résultats de structure sur les applications
affines (autrement dit pour pouvoir en simplifier l'expression en les
écrivant en fonction d'autres coordonnées) on doit pouvoir passer
d'une base à une autre.
\begin{prop}
  Soient $\bs{v}$ et $\bs{v'}$ deux bases de $\R^n$, $\bs{w}$ et
  $\bs{w'}$ deux bases de $\R^m$, $M$, $b$ les données d'une écriture
  d'une application affine $f$ dans les bases $\bs{v}$ et $\bs{w}$, et
  $M'$ et $b'$ les données de cette de l'écriture de cette même
  application dans les bases $\bs{w}$ et $\bs{w'}$. Si
  $P_{\bs{v'}}^{\bs{v}}$ est la matrice des vecteurs de $\bs{v}$ dans
  la base $\bs{v'}$ et $Q_{\bs{w'}}^{\bs{w}}$ celle des vecteurs de
  $\bs{w}$ dans la base $\bs{w'}$ alors pour tout $x \in \R^n$
  \[
  M'x_{\bs{v'}} + b'_{\bs{w'}} = f(x) =
  P_{\bs{w'}}^{\bs{w}}M\left(P_{\bs{v'}}^{\bs{v}}\right)^{-1}x_{\bs{v}}
  + P_{\bs{w'}}^{\bs{w}}b_{\bs{w}}
  \]
\end{prop}
\begin{question}
  Écrire la fonction du premier exemple dans la base de $\R^2$ au
  départ et à l'arrivée donnée par la base $\{(1, 1), (1, -1)\}$.
\end{question}
\begin{rem}
  Dans les faits, on n'aura pas souvent à faire ce type de changement
  de bases. À partir des propriété de notre application on sera à
  même de trouver un bonne base dans laquelle la décrire. Il reste que
  de tels changements de bases doivent être compris et faits quand
  nécessaire.
\end{rem}
\begin{exmp}
  Les applications affines constantes sont celles dont l'expression
  dans toute base a une matrice qui est nulle.
\end{exmp}
\begin{exmp}
  Avec la définition qu'on a donné toute application linéaire va
  décrire une application affine. L'expression de l'application affine
  dans des bases $\bs{v}$ et $\bs{w}$ des espaces de départ et
  d'arrivée est simplement donnée par la matrice de l'application
  linéaire dans les bases en question.
\end{exmp}
\begin{exmp}
  La translation de vecteur $b$ est l'application affine donnée dans
  toute base $\bs{v}$ de $\R^n$ par
  $x_{\bs{v}} \mapsto x_{\bs{v}} + b_{\bs{v}}$.
\end{exmp}
\begin{exmp}
  L'\emph{homothétie} de rapport $\lambda$ et centre $b$ est
  l'application affine de $\R^n$ dans lui-même dont l'écriture en
  toute base $\bs{v}$ prend la forme
  \[
  x_{\bs{v}} \mapsto \lambda (x_{\bs{v}}-b_{\bs{v}}) + b_{\bs{v}}
  \]
  L'homothétie de rapport $1$ est simplement l'identité.
\end{exmp}
\begin{question}
  Donner une particularité que partage les homothéties et les
  translations.
\end{question}
\begin{question}
  Que donne la composition de deux translations? Celle de deux
  homothéties?
\end{question}
\begin{question}
  Dessiner l'image du carré de sommets $(\pm 1, \pm 1)$
  dans $\R^2$ par les homothéties suivantes:
  \begin{itemize}
  \item[\textbullet]
    de rapport $2$ et de centre l'origine
  \item[\textbullet]
    de rapport $2$ et de centre $(1, 0)$.
  \end{itemize}
\end{question}

\subsection{Projections Affines}

Les projections affines sont la généralisation des projections
linéaires. Pour rappel une projection $p : E \rightarrow E$ de
l'espace vectoriel $E$ sur lui même est un application linéaire qui
satisfait la relation $p^2 = p$. Cette relation garanti le fait que
$G = \Ker(p)$ et $F = \Ker(p-\id)$ sont des espaces vectoriels
supplémentaires dans $E$. Ainsi pour tout $x = x_F + x_G$ la
projection $p$ est définie par $p(x) = x_F$. L'application $p$ est
appelée projection sur $F$ parallèlement à $G$.
\begin{defn}
  Une application affine $f$ de $\R^n$ dans $\R^n$ est une projection
  affine si sa matrice dans une base quelconque est la matrice d'une
  projection linéaire.
\end{defn}
Plus précisément une projection affine s'écrit dans une base $\bs{v}$
de $\R^n$ sous la forme
\[
x_{\bs{v}} \mapsto M(x_{\bs{v}} - b_{\bs{v}}) + b_{\bs{v}}
\]
On appelle parfois $b_{\bs{v}}$ le centre de la projection $f$. Dans ce
contexte $f$ est la projection de $b_{\bs{v}} + \Ker(M)$ parallèlement
à $b_{\bs{v}} + \Ker(M-I_n)$.

\begin{question}
  Étudier des exemples de projections affines dans $\R^2$. Dessiner
  systématiquement les droites affines qui caractérisent cette
  projection.
\end{question}

\pretitle{\vspace{-2\baselineskip} \begin{center}}
\title{%
  { \huge Solutions des exercices}%
}
\posttitle{
\end{center}
  \vspace{.5\baselineskip}
  \rule{\textwidth}{1.5pt}
  \vspace{-5\baselineskip}
}

\maketitle\thispagestyle{fancy}

\noindent Vous trouverez dans la suite solutions et indications d'une
partie des exercices de la feuille. Ceci étant majoritairement
accessibles il vous est suffisant de comparer votre travail au
résultats que vous retrouverez dans la suite.

\end{document}

%%% Local Variables:
%%% mode: latex
%%% TeX-master: t
%%% End:
