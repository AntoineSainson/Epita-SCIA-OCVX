\documentclass[11pt, a4paper]{article}

\usepackage[french]{babel}
\usepackage{fancyhdr}
\usepackage[margin=.8in]{geometry}

\usepackage{Style/TeXingStyle}

\pagestyle{fancy}
\renewcommand{\headrulewidth}{1.5pt}
\renewcommand{\footrulewidth}{0.5pt}
\fancyhead[L]{EPITA\_ING2\_2018\_S8}
\fancyhead[R]{Majeure SCIA}
\fancyhead[C]{OCVX1}
\fancyfoot[C]{\thepage}
\fancyfoot[L]{Juin 2017}
\fancyfoot[R]{\textbf{Chargé de cours :} \textsc{Bashar~DUDIN}}

\pretitle{\vspace{-.5\baselineskip} \begin{center}}
\title{%
  { \huge Examen d'optimisation convexe}%
}
\posttitle{
\end{center}
  \begin{flushleft}
    \vspace{3\baselineskip} \textit{
      \!\!\emph{Durée de l'épreuve 3h.}\\
      \! \emph{Les documents du cours ainsi que les calculettes ne sont pas
      autorisées.}  \\
      Le barême est indicatif et peut évoluer de manière marginale. Il
      contient un maximum de $28$ points non bonus, les notes seront
      rapportées à une note inférieure, probablement $20$.
    }
  \end{flushleft}
  \rule{\textwidth}{1.5pt}
  \vspace{-5\baselineskip}
}
\author{}
\date{}

\pdfinfo{
   /Author (Bashar Dudin)
   /Title  (Examen optimisation convexe - 2017)
   /Subject (Optimisation convexe)
}

\begin{document}

\maketitle\thispagestyle{fancy}

\section{Calcul différentiel élémentaire}

À moins qu'on y fasse explicitement référence, il n'est pas nécessaire
de justifier la différentiabilité des fonctions que vous
manipulez. \emph{Vous ne devez résoudre que deux questions parmis les
  trois de cette section!} À default, uniquement les deux premières
seront corrigées.

\begin{question}{2}
  Expliciter la jacobienne, en tout point, des expressions
  différentiables suivantes :
  \begin{enumerate}
  \item $f(x, y) = \big(y\sin(x), \cos(x)\big)$
  \item $g(x, y, z) = \dfrac{x + y^2 + xyz}{x^2+1}$
  \end{enumerate}
\end{question}

\begin{question}{3}
  Calculer la différentielle des fonctions suivantes:
  \begin{enumerate}
  \item $f : M_n(\R) \rightarrow \R$ donnée par l'expressoin
    $A \mapsto A^TA - \mathrm{tr}(A)$, où $I$ désigne la matrice
    identité de $M_n(\R)$ et $\mathrm{tr}$ la trace.\footnote{La somme
      des coefficients diagonaux d'une matrice.}
  \item $f : M_n(\R) \rightarrow M_n(\R)$ donnée par l'expressoin
    $A \mapsto \mathrm{tr}(A)A$.
  \end{enumerate}
\end{question}

\begin{question}{2+1}
  On considère la fonction définie sur $\R^2$ par
  $f(x, y) = x\cos(y) + y\exp(x)$. Pour tout $\theta \in [0, 2\pi[$ on
  désigne par $v_\theta$ le vecteur
  $\big(\cos(\theta), \sin(\theta)\big)$.
  \begin{enumerate}
  \item Calculer la dérivée directionnelle $\pa_{v_\theta}f(0, 0)$ en fonction de $\theta$.
  \item Pour quelles $\theta$ la dérivée directionnelle précédente est maximale (resp. minimale) ?
  \item Comment interpréter ce résultat géométriquement ?
  \end{enumerate}
\end{question}

\section{Problèmes d'optimisation}

\subsection{Un programme linéaire en petite dimension}

\noindent On considère le programme linéaire (\emph{PL}) suivant :
\begin{displaymath}
  \begin{linearProg} {
      minimiser
    }{
      $f_0(x_1, x_2) = \dfrac{x_1}{2} + x_2$
    }{
      \systeme{
        -x_1 - x_2 \leq 1,
        -4x_1 + x_2 \leq -4,
        -x_1 + 2x_2 \leq 2
      }
    }
  \end{linearProg}
\end{displaymath}
On va chercher dans la suite à étudier (\emph{PL}) : vous aurez à le
résoudre puis à vous en servir comme base de départ pour des exemples
de comportements spécifiques de programmes linéaires.
\begin{question}{5 + 1}
  \begin{enumerate}
  \item Représenter le lieu admissible de (\emph{PL}) dans le plan
    euclidien.
  \item
    \begin{enumerate}
    \item[a.]  Tracer la courbe de niveau $4$ de la fonction objectif
      de (\emph{PL}). Elle sera notée $C_4$.
    \item[b.]  Indiquer les demi-espaces positif et négatif défini
      par $C_4$.
    \item[c.]  Indiquer dans quelle direction on doit translater $C_4$
      afin de minimiser $f_0$. Quelle est le lien avec le gradient de
      $f_0$?
    \end{enumerate}
  \item
    Tracer la courbe de niveau qui réalise le minimum de (\emph{PL})
    et calculer l'unique point optimal de (\emph{PL}). Quelle est la
    valeur optimale de (\emph{PL})?
  \item Trouver des programmes linéaires ayant mêmes lieux admissibles
    que (\emph{PL}) et qui satisfont, à tour de rôle, chacune des
    condition suivantes :
    \begin{enumerate}
    \item[a.] être non borné ;
    \item[b.] avoir une infinité de points optimaux.
    \end{enumerate}
    Par quelle condition sur le gradient de $f_0$ peut-on décrire
    toutes les situations où l'on a une infinité de points optimaux?
  \end{enumerate}
\end{question}

\subsection{Baby example}

L'approche géométrique dans le cas des programmes linéaires de petites
dimensions s'étend à certains problèmes d'optimisations simples.

\begin{question}{6}
On considère la fonction différentiable $f : \R^2 \to \R$ donnée par
\[
f(x, y) = 2x^2 + y^2 - 1.
\]
\begin{enumerate}
\item Représenter les courbes de niveaux $0$, $1$, et $3$ de $f$
  dans le plan euclidien.
\item À quel lieu correspond la condition $f(x, y) \leq 3$ ?
\item On s'intéresse au problème d'optimisation (\emph{P1})
  \[
  \begin{PbOptim}{
      minimiser
    }{
      $x - y$
    }{
      $2x^2 + y^2 - 1 \leq 3$
    }
  \end{PbOptim}
  \]
  Représenter la courbe de niveau de la fonction objective qui
  correspond à la valeur optimale de (\emph{P1}).
\item Comment trouver le point optimale correspondant à (\emph{P1}) ?
  Faites le calcul.
\item Expliciter le problème dual (\emph{P1D}).
\item Résoudre (\emph{P1D}) et comparer sa solution avec celle de
  (\emph{P1}).
\end{enumerate}
\end{question}

\subsection{Optimisation sans contraintes}

Le problème des moindres carrés est un problème d'optimisation
(\emph{P2}) qui prend la forme
\[
\textrm{minimiser  } \frac{1}{2}\| A\theta - y\|_2^2
\]
où $A$ est une matrice dans $M_{n, m}(\R)$, $y \in \R^n$ et $\theta$
est un vecteur de $m$ paramètres à determiner. Le vecteur $y$ est
pensé comme la variable à expliquer. Le vecteur $\theta$ correspond
aux paramètres de la fonction affine qui explique $y$ suivant les
lignes de $A$.
\begin{question}{3}
  On suppose que les colonnes de $A$ sont indépendantes, cela implique
  que $A^TA$ est inversible.
  \begin{enumerate}
  \item Justifier le fait que (\emph{P2}) est un problème
    d'optimisation convexe. Vous êtes autorisés à utiliser des
    résultats du devoir maison.
  \item Résoudre le problème d'optimisation (\emph{P2}).
  \end{enumerate}
\end{question}
Sous les mêmes hypothèses précédentes on s'intéresse désormais au
problème d'optimisation (\emph{P2'})
\[
\textrm{minimiser } \frac{1}{2}\| A\theta - y\|_2^2 +
\frac{\nu}{2}\|\theta\|^2
\]
avec $\nu \geq 0$.
\begin{question}{4+1}
  \begin{enumerate}
  \item Justifier le fait que (\emph{P2'}) est un problème
    d'optimisation convexe.
  \item Écrire la condition d'annulation du gradient de la fonction
    objective de (\emph{P2'}).
  \item Donner une condition sous laquelle on peut résoudre
    (\emph{P2'}).
  \item Qu'apporte le problème (\emph{P2'}) par rapport à (\emph{P2}),
    notamment quand $\nu$ est assez grand?
  \end{enumerate}
\end{question}

\subsection{Conditions KKT}

Une entreprise fabrique deux modèles de clés USB, les modèles $X$ et
$Y$. Le modèle $X$ se vend à $1$\euro{} pièce, le modèle $Y$ se vend pour
sa part à $3$\euro{} pièce. Le coût de fabrication en \euro{} est donné par
l'expression :
\[
C(x, y) = 5x^2 + 5y^2 - 2xy - 2x - 1000.
\]
où $x$ est le nombre de clés USB de type $X$ et $y$ celui de type
$Y$. On suppose que toutes les clés USB ont été toutes écoulées sur le
marché.

\begin{question}{6}
  \begin{enumerate}
  \item Soit $(x, y) \in (\R_+^*)^2$. Déterminer le profit $P(x, y)$
    réalisé par l’entreprise lorsqu’elle a vendu $x$ clés de modèle $X$
    et $y$ clés de modèle $Y$.
  \item Étudier la convexité de la fonction de profit $P$ sur
    $(\R_+^*)^2$. Vous pouvez vous aider de la hessienne de $P$ en
    tout point.
  \item La capacité de production de l’entreprise est au total de $20$
    clés par jour\footnote{C'est des clés fait main!}. Décrire le
    problème de maximisation du profit de l'entreprise comme un
    problème d'optimisation convexe.
  \item En supposant que l’entreprise tourne à plein régime, trouver
    la répartition optimale entre les modèles de type $X$ et $Y$
    permettant de maximiser le profit quotidien. Calculer dans ce cas
    le profit réalisé. \textit{Vous êtes invités, en un premier temps,
      à ne pas tenir compte des contraintes de positivité de $x$ et
      $y$.}
  \item Pourquoi peut-on se dispenser des contraintes de positivité ?
  \end{enumerate}
\end{question}



\end{document}

%%% Local Variables:
%%% mode: latex
%%% TeX-master: t
%%% End:
