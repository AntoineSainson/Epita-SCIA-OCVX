\documentclass[11pt, a4paper]{article}

\usepackage[french]{babel}
\usepackage{fancyhdr}
\usepackage[margin=.8in]{geometry}

\usepackage{Style/TeXingStyle}

\pagestyle{fancy}
\renewcommand{\headrulewidth}{1.5pt}
\renewcommand{\footrulewidth}{0.5pt}
\fancyhead[L]{EPITA\_ING2\_2019\_S8}
\fancyhead[R]{Majeure SCIA}
\fancyhead[C]{OCVX1}
\fancyfoot[C]{\thepage}
\fancyfoot[L]{Avril 2018}
\fancyfoot[R]{\textbf{Chargé de cours :} \textsc{Bashar~DUDIN}}

\pretitle{\vspace{-.5\baselineskip} \begin{center}}
\title{%
  { \huge Examen d'optimisation convexe}%
}
\posttitle{
\end{center}
  \begin{flushleft}
    \vspace{3\baselineskip} \textit{
      \!\!\emph{Durée de l'épreuve 3h.}\\
      \! \emph{Les documents du cours ainsi que les calculettes ne sont pas
      autorisées.}  \\
      Le barême est indicatif et peut évoluer de manière marginale. Il
      contient un maximum de $36$ points bonus compris, les notes seront
      rapportées à une note inférieure, probablement $20$.
    }
  \end{flushleft}
  \rule{\textwidth}{1.5pt}
  \vspace{-5\baselineskip}
}
\author{}
\date{}

\pdfinfo{
   /Author (Bashar Dudin)
   /Title  (Examen optimisation convexe - 2019)
   /Subject (Optimisation convexe)
}

\begin{document}

\maketitle\thispagestyle{fancy}

\section{Géométrie élémentaire dans $\R^2$}
\label{sec:geom}

Dans cette section je cherche à vous faire représenter des lieux de
$\R^2$ que vous rencontrerez régulièrement. Elle permettra d'évaluer
également les différents éléments de terminologie introduit en cours.

\noindent On désigne par $A$ la partie de $\R^2$ donnée par
\[
A = \left\{ (x, y) \in \R^2 \; \left| \; \begin{matrix} x + 2y & \leq 3 \\ x - y &
      \geq 2 \end{matrix}\right.\right\}.
\]
\begin{question}{4}
  \begin{enumerate}
  \item
    Représenter $A$ graphiquement en indiquant les éléments qui
    permettent de décomposer votre représentation.
  \item
    Quel est le lieu qu'on obtient si on inverse les inégalités.
  \item
    Donner l'équation d'un demi-espace dont l'intersection avec $A$
    est un lieu non-vide borné de $\R^2$.
  \end{enumerate}
\end{question}
On note désormais $B$ l'intersection de l'épigraphe de la fonction
$x \mapsto -\sqrt{x}$ sur $\R_+$ et de la partie
\[
\left\{(x, y) \mid y \leq \sqrt{x}\right\}.
\]
\begin{question}{4}
  \begin{enumerate}
  \item Représenter $B$ graphiquement.
  \item Rappeler la définition de convexité d'une partie.
  \item Donner deux arguments qui permettent de justifier le fait que
    $B$ est convexe.
  \end{enumerate}
\end{question}
On considère les fonctions suivantes notées $f$, $g$ et $h$
respectivement données par les expressions
\[
  f(x, y) = x + 4y, \quad g(x, y) = xy, \quad h(x, y) = \frac{x^2}{2} + \frac{y^2}{4}.
\]
\begin{question}{3 + 2}
  \begin{enumerate}
  \item Représenter les courbes de niveaux $\mc{C}_1{f}$,
    $\mc{C}_2(g)$ et $\mc{C}_{4}(h)$.
  \item Donner un hyperplan d'appui à $\mc{C}_{\leq 4}(h)$.
  \item Justifier le fait que $\mc{C}_{\leq 2}(g)$ n'a pas d'hyperplan
    d'appui.
  \end{enumerate}
\end{question}
On s'intéresse pour terminer cette section à la convexité de la
fonction $x \mapsto x^2$ sur $\R$\footnote{Cet exercice est légèrement
  technique et vous pourrez le garder pour la suite.}.
\begin{question}{3}
  À l'aide de la définition de convexité d'une fonction, montrer que
  $x \mapsto x^2$ est une fonction convexe sur $\R$.
\end{question}


\section{Calcul différentiel élémentaire}

À moins qu'on y fasse explicitement référence, il n'est pas nécessaire
de justifier la différentiabilité des fonctions que vous
manipulez.
\begin{question}{1}
  Calculer la différentielle de l'application sur $\R$,
  $f(x) = \frac{\sin(x)}{x^2+1}$ en tout point $x$. Quelle est la
  différence avec la dérivée de $f$.
\end{question}
On se donne une matrice $A \in \mc{M}_n(\R)$.
\begin{question}{5}
  Calculer la différentielle des fonctions suivantes:
  \begin{enumerate}
  \item $f :\R^n \rightarrow \R$ donnée par l'expressoin
    $X \mapsto X^TAX$.
  \item $h : \mc{M}_n(\R) \rightarrow \mc{M}_n(\R)$ donnée par
    l'expression $B \mapsto \mathrm{tr}(AB)B$.
  \item $g : \R^n \rightarrow \R$ donnée par l'expression
    $X \mapsto 1/(X^TX + 1)$.
  \item $k :\R^n \rightarrow \R$ donnée par l'expressoin
    $X \mapsto \sin(X^TAX)$.
  \end{enumerate}
\end{question}
\begin{question}{3}
  Expliciter le gradient, en tout point, des expressions
  différentiables suivantes :
  \begin{enumerate}
  \item $f(x, y, z) = \dfrac{x + y + z}{x+y^2+1}$ ;
  \item $g(x, y) = \dfrac{\cos(xy)}{\sqrt{x^2 + y^2}}$ ;
  \item $h(x, y, z) = \exp(xy - z)$.
  \end{enumerate}
\end{question}

\section{Problèmes d'optimisation simples}

\subsection{De l'existence de points optimaux}

On considère les contraintes linéares sur $\R^2$
\[
\systeme{
  x + 2y \leq 3,
  x - y  \geq 2
}
\]
décrivant la partie $A$ section \ref{sec:geom}.
\begin{question}{1.5}
  Donner pour chacune des propriétés suivantes un programme linéaire
  ayant $A$ pour lieu admissible\footnote{Lieu décrit par les
    contraintes du programme linéare.} et satisfaisant cette propriété
  \begin{enumerate}
  \item le programme linéaire n'a qu'un seul point optimal ;
  \item le programme linéaire en a une infinité;
  \item le programme linéaire n'est pas borné.
  \end{enumerate}
\end{question}

\subsection{Un programme linéaire en petite dimension}

\noindent On considère le programme linéaire (\emph{PL}) suivant :
\begin{displaymath}
  \begin{linearProg} {
      minimiser
    }{
      $x + 2y$
    }{
      \systeme{
        x + y \leq 1,
        -x + 2y \leq 2,
        x - 3y \leq 3
      }
    }
  \end{linearProg}
\end{displaymath}
On va chercher dans la suite à étudier (\emph{PL}).
\begin{question}{4}
  \begin{enumerate}
  \item Représenter le lieu admissible de (\emph{PL}) dans $\R^2$.
  \item
    \begin{enumerate}
    \item[a.]  Tracer la courbe de niveau $0$ de la fonction objectif
      de (\emph{PL}). Elle sera notée $\mc{C}_0$.
    \item[b.]  Indiquer les demi-espaces positif et négatif définis
      par $\mc{C}_0$.
    \item[c.]  Indiquer dans quelle direction on doit translater
      $\mc{C}_0$ afin de minimiser la fonction objective.
    \end{enumerate}
  \item
    Tracer la courbe de niveau qui réalise le minimum de (\emph{PL})
    et calculer l'unique point optimal de (\emph{PL}). Quelle est la
    valeur optimale de (\emph{PL})?
  \end{enumerate}
\end{question}

\subsection{Baby example}

L'approche géométrique dans le cas des programmes linéaires de petites
dimensions s'étend à certains problèmes d'optimisations simples. On
considère le probème d'optimisation (\emph{P}) suivant
\[
\begin{PbOptim}{
    minimiser
  }{
    $x + y$
  }{
    $\begin{matrix}
      x+2y \leq 3 \\
      x \in B
    \end{matrix}$
  }
\end{PbOptim}
\]
où $B$ est la partie de $\R^2$ qu'on a décrit section \ref{sec:geom}.

\begin{question}{1.5 + 3}
  \begin{enumerate}
  \item Dessiner le lieu admissible du problème de (\emph{P}).
  \item Représenter la courbe de niveau de la fonction objective qui
    réalise le minimum de (\emph{P})\footnote{croquis de son
      positionnement.}.
  \item Calculer le point optimal ainsi que la valeur optimale de
    (\emph{P}).
  \end{enumerate}
\end{question}

\end{document}

%%% Local Variables:
%%% mode: latex
%%% TeX-master: t
%%% End:
