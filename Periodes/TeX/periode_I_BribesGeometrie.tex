\documentclass[11pt, a4paper]{article}

\usepackage[french]{babel}
\usepackage{fancyhdr}
\usepackage[margin=.8in]{geometry}

\usepackage{Style/TeXingStyle}

\pagestyle{fancy}
\renewcommand{\headrulewidth}{1.5pt}
\renewcommand{\footrulewidth}{0.5pt}
\fancyhead[L]{EPITA\_ING2\_2020\_S8}
\fancyhead[R]{Majeures SCIA \& IMAGE}
\fancyhead[C]{OCVX}
\fancyfoot[C]{\thepage}
\fancyfoot[L]{2019}
\fancyfoot[R]{\footnotesize{\textbf{Chargés de cours :} \textsc{B.~DUDIN} \& \textsc{G.~TOCHON}}}

\pretitle{\vspace{-2.5\baselineskip} \begin{center}}
\title{%
  { \huge Bribes de géométrie}%
}
\posttitle{
\end{center}
\rule{\textwidth}{1.5pt}
\vspace{-3\baselineskip}
}
\author{}
\date{}

\pdfinfo{
   /Author (Bashar Dudin)
   /Title  (Bribes de géométrie - 2019)
   /Subject (SCIA/IMAGE - Optimisation convexe)
}

\setlength\parindent{0pt}

\begin{document}

\maketitle\thispagestyle{fancy}

\begin{abstract}
  Cette première série de séances pédagogiques se focalise sur les
  éléments de géométrie et, dans une certaine mesure, de topologie
  sous-jacents aux différents contextes qu'on aborde dans la
  suite.
\end{abstract}

\tableofcontents

\newpage

\begin{center}
 \begin{minipage}{.6\textwidth}
    \begin{convention}
      On fixe dans la suite un entier naturel $n$ strictement positif. Le
      cas nul, même s'il garderait un sens dans les énoncés a suivre, nous
      est pas d'un grand intérêt. Afin de ne pas alourdir le texte, la
      $i$-ème coordonnée d'un vecteur $x \in \R^n$ sera
      systématiquement notée $x_i$. La transposée d'une matrice $A$ sera
      notée $A^T$.
    \end{convention}
  \end{minipage}
\end{center}

\section{Structure euclidienne usuelle sur $\R^n$}

Un espace euclidien est un espace vectoriel muni d'un produit
scalaire, c'est-à-dire d'une forme bilinéaire symétrique définie
positive. On vous renvoie à la section (\ref{sec:FBS}) pour un
rafraîchissement de mémoire. L'espace vectoriel $\R^n$ vient
muni d'une structure euclidienne, dite usuelle, définie par le produit
scalaire suivant: à tout couple de vecteurs $(x, y)$ dans
$\R^n$ on associe le scalaire
\[
\langle x, y \rangle = \sum_{i=1}^n x_iy_i.
\]
Cette quantité est souvent exprimée de manière plus compacte par
\[
\langle x, y \rangle = x^Ty.
\]
La longueur au sens intuitif du vecteur $x$, appelée norme euclidienne
de $x$, s'exprime alors sous la forme
\[
  \|x\|_2 = \langle x, x\rangle^{1/2} = \sum_{i=1}^n x_i^2.
\]
En ce sens la norme $\|\cdot\|_2$ est dite \emph{norme associée au
  produit scalaire} $\langle \cdot\,, \cdot \rangle$. On rappelle que
celle-ci permet d'écrire l'inégalité de \tsc{Cauchy}-\tsc{Schwarz}
sous la forme
\[
\big| \langle x, y\rangle \big| \leq \|x\|_2\|y\|_2
\]
où $x$ et $y$ sont des vecteurs quelconques de $\R^n$.

\subsection{Produit scalaire de deux vecteurs}

Le signe du produit scalaire de deux vecteurs porte une signification
géométrique dont la compréhension est essentielle à l'étude
qualitative des problèmes d'optimisation, ainsi qu'aux méthodes
itératives qui permettent d'en approcher une solution.

On se place en un premier temps dans le cas de dimension $2$, du plan
euclidien. Soient $x$ et $y$ deux vecteurs de $\R^2$, on
désigne par $\theta$ l'angle orienté (dans le sens direct) entre $x$
et $y$ et par $\phi$ (resp. $\psi$) celui entre $x$ (resp. $y$) et la
partie positive de l'axe des abscisses.
\begin{question}
  \begin{enumerate}
  \item Exprimer les coordonnées de $x$ et $y$ en fonction de leurs
    normes respectives et des angles $\phi$ et $\psi$.
  \item En déduire une expression du produit scalaire de
    $\langle x, y \rangle$ en fonction de $\theta$ et des normes de $x$
    et $y$.
  \end{enumerate}
\end{question}
\begin{solution}
  De vagues souvenirs de
  physique devraient vous revenir, les coordonnées de $x$ et $y$ dans la
  base canonique de $\R^2$ s'expriment par
  \[
    \begin{matrix}
      x_1 = \|x\|_2\cos(\phi), & x_2 = \|x\|_2\sin(\phi), \\
      y_1 = \|y\|_2\cos(\psi), & y_2 = \|x\|_2\sin(\psi).
    \end{matrix}
  \]
  Donc
  \[
    \begin{array}{rcl}
      \langle x, y \rangle & = & x_1y_1 + x_2y_2 \\
                           & = & \|x\|_2\|y\|_2\big(\cos(\phi)\cos(\psi) + \sin(\phi)\sin(\psi)\big) \\
                           & = & \|x\|_2\|y\|_2\cos(\phi - \psi) \\ \\
      \langle x, y \rangle & = & \|x\|_2\|y\|_2\cos(\theta).
    \end{array}
  \]
\end{solution}

L'étude précédente s'étend relativement facilement au cas $n \geq
2$. Soit $P$ un plan contenant les vecteurs $x$ et $y$ de
$\R^n$, si ceux-ci sont indépendants $P$ est bien entendu
l'espace vectoriel engendre par $x$ et $y$. La restriction du produit
scalaire $\langle \cdot\, , \cdot\rangle$ à $P$ prend la forme usuelle
dans toute base orthonormée de $P$. Pour en trouver une dans $P$ il
suffit de normaliser un vecteur non nul $f_1$ de $P$ puis de
normaliser un vecteur $f_2$ de son orthogonal dans $P$. C'est la
première étape de l'algorithme de \tsc{Gram}-\tsc{Schmidt}. La base
$(f_1, f_2)$ nous ramène au cas de $\R^2$ précédent.

Le travail précédent permet d'établir en particulier le fait que
l'orthogonalité de deux vecteurs, au sens intuitif que vous avez
<<\,\textit{l'angle entre les deux vecteur est égal à $\pi/2 [\pi]$}\,>>,
correspond à l'annulation de leur produit scalaire. C'est cette
définition, plus algébrique et n'impliquant pas de définir ce qu'est
un angle, qui est utilisée en pratique.
\begin{defn}
  Deux vecteurs $x$, $y$ d'un espace euclidien
  $(E, \langle \cdot, \cdot \rangle)$ sont dits \emph{orthogonaux} si
  leur produit scalaire est nul.
\end{defn}
Cette notion d'orthogonalité de deux vecteurs va également permettre
de décrire certains sous-espace d'un espace euclidien.
\begin{defn}
  Soit $(E, \langle \cdot, \cdot \rangle)$ un espace
  euclidien. L'\emph{orthogonal} d'un sous-espace vectoriel $F$ de $E$
  est le sous-espace vectoriel défini par
  \[
    F^\perp = \{x \in E \mid \forall y \in F, \langle x, y \rangle =
    0\}.
  \]
\end{defn}
Par abus on parlera d'orthogonal à un vecteur (ou une famille de
vecteur) quand on sous-entend l'orthogonal au sous-espace vectoriel
engendré. À noter que lorsque un vecteur est orthogonal à une famille
de vecteurs, il est également orthogonal à toute combinaison linéaire
d'éléments de celle-ci.

On revient à la signification géométrique du produit scalaire. On est
désormais en mesure de voir qu'étant donné un vecteur non nul
$\nu \in \R^2$, le lieu des vecteurs $x$ sujets à la contrainte
$\langle \nu, x \rangle = \nu^Tx \geq 0$ est celui des vecteurs $x$
tels que $\measuredangle x\nu$ est aigu (ou orthogonal). C'est
précisément le demi-espace de $\R^2$ délimité par $\{\nu\}^\perp$ et
vers lequel pointe $\nu$. La contrainte $\nu^Tx \leq 0$ correspond de
son côté au demi-espace délimité par $\{\nu\}^{\perp}$ et hors duquel
pointe $\nu$. C'est l'ensemble des vecteurs $x$ qui décrivent un angle
$\measuredangle x\nu$ obtus.
\begin{question}
  \label{q:exmpdetravail}
  Décrire le lieu de $\R^2$ donné par la relation matricielle:
  \[
    \begin{pmatrix}
      1 & 2 \\
      -1 & 1 \\
    \end{pmatrix}
    \begin{matrix}
      x \\ y
    \end{matrix}
    \leq
    \begin{pmatrix}
      0 \\ 0
    \end{pmatrix}
  \]
\end{question}
Dans $\R^n$ un demi-espace n'est plus délimité par une droite
mais par un sous-espace vectoriel de dimension $n-1$, c'est ce qu'on
appelle un \emph{hyperplan} de $\R^n$. Si $\nu$ est un vecteur
non nul de $\R^n$, la condition de $\nu^Tx \geq 0$ décrit les
vecteurs $x$ dans le demi-espace délimité par l'orthogonal de $\nu$ et
vers lequel pointe $\nu$. Inverser l'inégalité revient à étudier le
demi-espace hors duquel pointe $\nu$.
\begin{question}
  Représenter le lieu de $\R^3$ décrit par la reations
  $x_1 + x_2 + x_3 \geq 0$.
\end{question}

\subsection{Dualité}

Si $E$ est un espace vectoriel on note $E^\vee$ l'espace vectoriel
$\mathcal{L}(E, \R)$ des formes linéaires sur $E$. C'est ce
qu'on appelle l'espace \emph{dual} de $E$, il est de même dimension
que $E$. Quand $E$ est euclidien, c'est-à-dire muni d'un produit
scalaire $\langle \cdot\, , \cdot\rangle$ et de dimension finie, on a
un isomorphisme d'espace vectoriel donné par l'application linéaire
naturelle
\[
\begin{array}{rll}
\Psi : E & \longrightarrow & E^\vee \\
x & \longmapsto & \big[\Phi_x : y \longmapsto \langle x, y \rangle\big]
\end{array}
\]
Pour s'en convaincre il suffit d'étudier le noyau de $\Psi$: si
celui-ci est trivial $\Psi$ est injective et donc bijective car $E$ et
$E^\vee$ ont même dimension, finie. Un vecteur $x$ est dans le noyau
de $\Psi$ si
\[
\forall y \in \R^2, \quad \langle x, y\rangle = 0.
\]
En particulier, on doit avoir $\langle x, x\rangle = 0$ donc $x = 0$.

Le résultat précédent signifie qu'étudier, par exemple, l'image d'une
sous-partie $A$ de $\R^n$ par toutes les formes linéaires de
son dual revient à regarder l'image de $A$ par les applications
linéaires $\Phi_x = \langle x\, , \cdot\rangle$ pour $x$ qui varie
dans $\R^n$. Cette seconde approche est plus explicite, elle
est donc plus évidente à implémenter (avec toutes les précautions
d'usage quant aux questions de discrétisation). Elle est également
loin d'être dénuée d'intérêt applicatif, on pourrait par exemple
s'intéresser à la trace que laisse un objet 3D par projections sur des
droites vectorielles de $\R^3$.
\section{Sous-espaces affines de $\R^n$}

Afin de rendre compte des situations géométriques auxquelles on fera
face, il va nous falloir élargir le panel des sous-ensembles de $\R^n$
que nous sommes capables de manipuler. La première étape est d'être en
mesure de travailler avec des translatés d'espaces vectoriels: les
sous-espaces affines de $\R^n$. On se limitera a une approche
utilitariste qui ne s'attarde pas sur la définition d'espace affine en
général (on vous renvoie vers l'annexe pour plus de formalisme). De ce
point de vue, de nombreux abus de terminologie seront fait.

\subsection{Définition}

\begin{defn}
  Un sous-espace affine de $\R^n$ est un translaté de sous-espace
  vectoriel de $\R^n$. Plus précisément, une partie $A$ de $\R^n$ est
  un sous-espace affine de $\R^n$ s'il existe un point $x$ dans $\R^n$
  et un sous-espace vectoriel $F$ de $\R^n$ tels que
\[
 A = x + F
\]
ou la notation $x + F$ désigne l'ensemble $\{x+y \mid y \in \R^n\}$.

Quand cette écriture a lieu, l'espace vectoriel $F$ est unique, il est
appelé la direction de $A$. Le point $x$ ne l'est cependant pas, voir
pour un exemple (\ref{q:affinesecritures}).
\end{defn}
La dimension d'un sous-espace affine de $\R^n$ est la
dimension de sa direction. En particulier, un hyperplan affine est le
translaté d'un hyperplan vectoriel de $\R^n$. Dans la pratique
on omet les adjectifs, le fait d'être vectoriel ou non est clair du
contexte.

\begin{exmp}
  Tout sous-espace vectoriel de $\R^n$ est un sous-espace affine de
  $\R^n$. En effet, on a toujours $F = 0 + F$.
\end{exmp}
\begin{exmp}
  Tout singleton $\{x\}$ de $\R^n$ définit un sous-espace affine de
  $\R^n$. Pour tout $x \in \R^n$ on peut toujours écrire
  $\{x\} = x + \{0\}$.
\end{exmp}
\begin{exmp}
  \label{exmp:espaceaffine}
  La partie $A = \{(1,t) \in \R^2 \mid t \in \R\}$ est un sous-espace
  affine de $\R^2$. On constate effectivement que
  \[
    A = (1, 0) + \{(0, t) \mid t \in \R\}
  \]
  le second terme étant évidemment un sous-espace vectoriel de
  $\R^2$. On peut par exemple remarquer que c'est le noyau de la
  projection $\pi_1 : \R^2 \rightarrow \R$ sur la première coordonnée
  (ou encore l'espace vectoriel engendré par $(0, 1)$). Donc
  \[
    A = (1, 0) + \Ker(\pi_1)
  \]
\end{exmp}
\begin{question}
  \label{q:affinesecritures}
  \begin{enumerate}
  \item Donner une écriture différente de l'espace affine $A$ dans l'exemple
    (\ref{exmp:espaceaffine}).
  \item Décrire toutes les écritures possibles de l'espace affine $A$ ;
    que pouvez vous en déduire en général?
  \end{enumerate}
\end{question}
\begin{solution}
  On peut également écrire $A$ sous la forme
  \[
    A = (1, 2) + \Ker(\pi_1).
  \]
  Toute écriture de $A$ prend la forme $x + \Ker(\pi_1)$ avec
  $x - (1, 0) \in \Ker(\pi_1)$. Cela vient du fait général suivant:
  écritures $x + F$ et $x'+ F$ sont égales si et seulement si $x - x'$
  soit dans $F$. On voit bien que dans notre cas particulier
  $(1, 2) - (1, 0) = (0, 2)$ est bien dans $\Ker(\pi_1)$.
\end{solution}
\begin{question}
  Décrire les sous-espaces affines de $\R$, $\R^2$ et $\R^3$.
\end{question}
\begin{solution}
  Dans $\R$ les sous-espaces affines sont les singletons et tout
  $\R$. Dans $\R^2$ on a les singletons, les droites
  affines et tout $\R^2$. Dans $\R^3$, on retrouve les
  singletons, les droites et les plans affines et enfin tout
  $\R^3$. La définition qu'on vient de donner rejoint donc les
  notions que vous avez du manipuler plus jeunes.
\end{solution}

\subsection{Se donner un sous-espace affine de $\R^3$}

Un sous-espace vectoriel de $\R^n$ est ou bien décrit par les
équations que vérifient ses éléments ou alors explicitement à l'aide
d'un certains nombres de paramètres. Dans le premier cas on dit que le
sous-espace vectoriel est donné implicitement, dans le second qu'il
est donné paramétriquement. Cela correspond à se donner un sous-espace
vectoriel ou bien comme le noyau ou alors comme l'image d'une
application linéaire. On peut encore dire qu'il est donné par
contrainte ou par le biais d'une base par laquelle l'espace vectoriel
est généré. La situation des sous-espaces affines est similaire.

\subsubsection{Paramétriquement}

Soit $P$ le sous-espace vectoriel de $\R^3$ donné parametriquement par
\[
P = \{(t, 3t+u, -u) \mid (t, u) \in \R^2\}
\]
C'est l'espace vectoriel engendré par la famille libre de $\R^3$,
$\{(1, 3, 0), (0, 1, -1)\}$.  C'est encore l'image de $\R^2$ par
l'application linéaire $(t, u) \mapsto (t, 3t+u, -u)$. Tout translaté
de $P$ est un sous-espace affine de $\R^3$, par exemple
\[
A = (1, 2, -1) + P = \{(1+t, 3t+u+2, -u-1)\mid (t, u) \in \R^2\}
\]
est un sous-espace affine de $\R^3$ donné paramétriquement. On voit
que la forme des coordonnées de $A$ est affine au sens que vous
entendiez plus jeunes. Une application affine est une application
linéaire une fois qu'on se débarrasse des constantes, plus
formellement
\begin{defn}
  Une application $f : \R^n \rightarrow \R^p$ est
  affine s'il existe $x_0 \in \R^p$ tel que
  $x \mapsto f(x) - x_0$ soit linéaire.
\end{defn}
Dans notre cas $x_0 = (1, 2, -1)$. Afin d'éviter un détour qui nous
écarterait de nos objectifs premiers nous garderons l'approche
intuitive de la notion d'application affines: une applications dont
l'expression des coordonnées est affines dans les variables de
départ. Pour plus détail on vous renvoie à (\ref{}).
\begin{question}
  Écrire paramétriquement:
  \begin{enumerate}
  \item la droite de $\R^2$ de vecteur directeur $(1, -1)$ et passant
    par $(2, 3)$ ;
  \item le plan de $\R^3$ donné par l'équation $x_1 + x_2 + x_3 = 2$.
  \end{enumerate}
\end{question}

\subsubsection{Implicitement}

Un sous-espace vectoriel $F$ de $\R^n$ est donne implicitement
s'il est décrit comme le noyau d'une application linéaire. Soit $M$
une matrice de taille $(p, n)$ et de coefficients $(m_{i,j})$. Un
vecteur $x \in \R^n$ est dans le noyau de l'application
linéaire associée à $M$ si
\[
Mx = 0
\]
relation compacte pour designer le système linéaire
\[
\left\{\begin{array}{cccccccc}
m_{1, 1} x_1 & + & \cdots & + & m_{1, n}x_n & = & 0 \\
m_{2, 1} x_1 & + & \cdots & + & m_{2, n}x_n & = & 0 \\
\vdots & & & & \vdots & \vdots & \vdots \\
m_{p, 1}x_1 & + & \cdots & + & m_{p, n}x_n & = & 0.
\end{array}\right.
\]
Un sous-espace affine de $\R^n$ est de manière similaire donné
par un système linéaire avec un second membre non nul, qui prend la
forme
\[
Mx = R \Longleftrightarrow \left\{\begin{array}{cccccccc}
m_{1, 1} x_1 & + & \cdots & + & m_{1, n}x_n & = & r_1 \\
m_{2, 1} x_1 & + & \cdots & + & m_{2, n}x_n & = & r_2 \\
\vdots & & & & \vdots & \vdots & \vdots \\
m_{p, 1}x_1 & + & \cdots & + & m_{p, n}x_n & = & r_3.
\end{array}\right.
\]
où $R$ désigne un vecteur de $\R^n$. Pour s'en convaincre on
se donne une solution particulière $x_0$ de $Mx = R$. L'équation
précédente est donc équivalente a
\[
Mx = Mx_0 \Longleftrightarrow M(x-x_0) = 0.
\]
L'ensemble des solutions de $Mx = R$ est donc ou bien vide ou alors
$x_0 + \Ker(M)$ pour une solution particul'nière $x_0$. Une fois donné
une base de $\Ker(M)$ on obtient une écriture paramétrique de $F$.
\begin{rem}
  Dans la pratique, on n'a pas à trouver d'abord une solution
  particulière de notre système afin d'en trouver toutes les
  solutions. Le pivot de Gauss s'en charge, vous êtes invités à vous
  rafraîchir la mémoire avec l'annexe (\ref{sec:pivotGauss}).
\end{rem}
\begin{question}
  Dessiner le lieu de $\R^2$ décrit par les contraintes
  \[
    \begin{pmatrix}
      -1 & 2 \\
      1 & 1
    \end{pmatrix}
    \begin{pmatrix}
      x \\ y
    \end{pmatrix}
    \leq
    \begin{pmatrix}
      -1 \\ 1
    \end{pmatrix}.
  \]
  \begin{itemize}
  \item Décrire chacun des composants du lieu géométrique précédent
    paramétriquement.
  \item Que change le fait de rajouter la contrainte $x - 3 y \leq 6$?
  \item Quel lieu correspond à la situation où l'on change le sens de
    toutes les inégalités?
  \end{itemize}
\end{question}
\begin{question}
  On désigne par $A$ la partie de $\R^2$ donnée par\footnote{Question
    1-1 Exam 2019.}
  \[
    A = \left\{ (x, y) \in \R^2 \; \left| \; \begin{matrix} x + 2y & \leq 3 \\ x - y &
          \geq 2 \end{matrix}\right.\right\}.
  \]
  \begin{enumerate}
  \item
    Représenter $A$ graphiquement en indiquant les éléments qui
    permettent de décomposer votre représentation.
  \item
    Quel est le lieu qu'on obtient si on inverse les inégalités.
  \item
    Donner l'équation d'un demi-espace dont l'intersection avec $A$
    est un lieu non-vide borné de $\R^2$.
  \end{enumerate}
\end{question}

\subsection{Cas extrémaux}

On a toujours le choix entre exprimer un sous-espace affine
paramétriquement ou implicitement. Il est cependant plus simple
d'exprimer une droite paramétriquement (on a besoin que d'un paramètre
pour le faire) et un hyperplan implicitement (une seule équation y
suffit).
\begin{itemize}
\item On pense une droite comme un point de départ auquel on ajoute
  les multiples scalaires d'un vecteur. On peut par exemple écrire
\[
  D = \{(0,1) + \lambda(2, 3) \mid \lambda \in \R\}
\]
Dans la pratique on verra souvent une écriture équivalente où l'on
pense une droite comme l'unique droite passant par deux points
donnés, dans le cas de $D$ on peut écrire
\[
  D = \{(1-t)(0, 1) + t(2, 4) \mid t \in \R\}
\]
\item Un hyperplan affine $H$ est pour sa part donné par une équation
  de la forme $\nu^Tx = r$. Si $x_0$ est une solution particulière de
  l'équation précédente celle-ci est équivalente à
\[
  \nu^T(x - x_0) = 0
\]
Donc $H = x_0 + \{\nu\}^{\perp}$. C'est le translaté de l'orthogonal
de $\nu$ par $x_0$.
\end{itemize}
\begin{question}
  Retrouver une écriture implicite de la droite $D$
  précédente. Comment s'y prendre pour transformer une écriture
  paramétrique en une écriture implicite?\footnote{Vous êtes invités à
    étudier plus en profondeur le passage représentation implicite
    $\to$ représentation paramétriques et inversement}
\end{question}
\section{Sous-espaces de $\R^n$}

On ne verra dans le cadre de ce cours qu'un contexte restreint de la
géométrie des parties de $\R^n$. On se limite par exemple aux parties
de $\R^n$ qu'on se donne à l'aide de fonctions de $\R^n$ vers $\R$. De
plus, dans le cas des descriptions paramétriques on se limite aux
graphes.

\subsection{Se donner des parties de $\R^n$}

Tout comme dans le cas des sous-espaces affines de $\R^n$, on peut se
donner une partie de $\R^n$ de manière \emph{implicite} ou
\emph{paramétrique}.

Une partie de $\R^n$ est décrite \emph{implicitement} lorsqu'elle est
donnée comme zéros de fonctions. Dans notre cadre, une telle partie
$A$ est de la forme
\[
  A = \{ x \in \R^n \mid f(x) = 0\}
\]
où $f$ est une fonction $f : \R^n \to \R$. Notez que cette écriture
regroupe les parties décrites par les équations de la forme
$f(x) = \star$ pour une constante $\star \in \R$. Une telle partie est
zéros de la fonction $g = f - \star$.
\begin{defn}
  Étant donné une fonction $f : \R^n \to \R$ on appelle \emph{courbe
    de niveau $\bs{r}$} de $f$ la partie de $\R^n$ définie par
  \[
    \mc{C}_r = \{ x \in \R^n \mid f(x) = r\}.
  \]
\end{defn}
Dans la suite on s'intéresse également au lieu de sous-niveau $r$
d'une fonction $f : \R^n \to \R$ ; c'est la partie de $\R^n$
\[
  \mc{C}_{\leq r} = \{ x \in \R^n \mid f(x) \leq r \}.
\]
\begin{question}
  Dessiner les courbes de niveaux $0$, $1$ et $2$ des fonctions
  $f : \R^2 \to \R$ et $g : \R^2 \to \R$ données par les expressions
  $f(x, y) = x^2 + y^2$ et $g(x, y) = x^2 + 4y^2$.
\end{question}
\begin{question}
  Comment décrire la surface délimitée par les deux branches d'une
  hyperbole?
\end{question}

On dira qu'une partie de $\R^{n+1}$ est décrite
\emph{paramétriquement} si elle prend la forme du graphe d'une
fonction
\[
  \Gamma(f) = \{ \big(t, f(t)\big) \mid t \in \R^n \}
\]
où $f : \R^n \to \R$. Une partie de $\R^{n+1}$ décrite
paramétriquement au sens précédent s'écrit facilement sous forme
implicite: c'est les zéros de la fonction
\[
  \begin{array}{rcl}
    \R^n \times \R & \longrightarrow & \R \\
    (x, y) & \longmapsto & f(x) - y
  \end{array}
\]
L'inverse n'est cependant pas vrai en général.
\begin{question}
  Trouver un exemple d'une partie de $\R^2$ qu'on peut décrire
  implicitement mais pas paramétriquement. Que faudrait-il modifier
  pour qu'il soit possible de parler d'écriture paramétrique dans
  votre exemple?
\end{question}
\begin{rem}
  Même avec une notion plus fine d'écriture paramétrique le passage
  implicite - paramétrique n'est pas toujours possible. Je vous invite
  à vous intérsser au théorème des fonctions implicites pour plus de
  détails.
\end{rem}
Les graphes de fonctions apparaissent régulièrement dans la suite ;
parmi les parties définies à partir de ceux-ci on retrouve l'épigraphe
d'une fonction $f : \R^n \to \R$
\[
  \mathrm{epi\,}(f) = \{ (x, t) \mid t \geq f(x) \}.
\]
\begin{question}
  Représenter l'épigraphe de la fonction $\sin$ sur
  $[0, \frac{\pi}{2}]$ puis sur $[\frac{\pi}{2}, \pi]$.
\end{question}
\begin{question}
  \label{q:partieB}
  Représenter graphiquement l'intersection de l'épigraphe de la
  fonction $x \mapsto -\sqrt{x}$ sur $\R_+$ et de la partie
  \[
    \left\{(x, y) \mid y \leq \sqrt{x}\right\}.
  \]
\end{question}
\begin{question}
  \label{q:courbeniveau}
  On considère les fonctions suivantes notées $f$, $g$ et $h$
  respectivement données par les expressions
  \[
    f(x, y) = x + 4y, \quad g(x, y) = xy, \quad h(x, y) = \frac{x^2}{2} + \frac{y^2}{4}.
  \]
  Représenter les courbes de niveaux $\mc{C}_1(f)$, $\mc{C}_2(g)$ et
  $\mc{C}_{4}(h)$.
\end{question}

\subsection{Convexité dans $\R^n$}

Les techniques d'optimisation se déroulent au mieux dans des contextes
convexes. C'est une notion qu'il est important de garder en mémoire ;
des comportements pathologiques voient le jour hors de ce cadre.

\subsubsection{Parties convexes de $\R^n$}
\label{sec:partiesconvexes}

\begin{defn}
  Une partie $A \subset \R^n$ est dite \emph{convexe} si l'on peut
  relier $2$ points quelconques de $A$ par un segment intégralement
  inclus dans $A$. Plus formellement : $A$ est convexe si
  \[
    \forall x, y \in A, \forall t \in [0, 1] \qquad (1-t)x + ty \in A.
  \]
\end{defn}
\begin{exmp}
  Tout intervalle de $\R$ est convexe.
\end{exmp}
\begin{exmp}
  Les sous-espaces affines de $\R^n$ sont convexes, il en va de même des
  demi-espaces de $\R^n$.
\end{exmp}
Dans la pratique reconnaître des parties convexes de $\R^n$ revient à
les construire à partir de parties convexes à l'aide d'opérations
préservant la convexité.
\begin{itemize}
\item L'intersection de parties convexes est convexe; le lieu
  admissible d'un programme linéaire est donc convexe.
\item L'image d'une partie convexe par une fonction affine est
  convexe; les changements de variables affines dans les contraintes
  d'un porgramme linéaire ne changent donc pas le caractère convexe de
  celui-ci.
\end{itemize}
\begin{question}
  Donner deux arguments qui permettent de justifier le fait que la
  partie décrite question (\ref{q:partieB}) est convexe.
\end{question}
\begin{question}
  Est-ce que l'union de parties convexes est convexe?
\end{question}
\begin{question}
  Qu'est-ce que l'enveloppe convexe d'une partie de $\R^n$?
\end{question}
Les parties convexes de $\R^n$ jouissent de propriétés géométriques
qui permettent de les isoler (séparer), lorsqu'elles ne s'intersectent
pas\footnote{Il faut un peu plus de rigueur, mais l'idée est là
  ...}. Ces problématiques sont en dehors de la portée de ce cours,
elles nous donnent cependant une information géométrique utile pour
généraliser l'étude des programmes linéaires au cas (convexe)
non-linéaire.
\begin{defn}
  Un point $x$ d'une partie $A \subset \R^n$ est \emph{point du bord}
  de $A$ si pour tout disque $D$ centré en $x$ et de rayon non nul
  $D \cap A \neq \emptyset$.

  L'ensemble des point du bord d'une partir $A$ est appelée
  \emph{bord} de $A$, elle est notée $\partial A$. L'union
  $\overline{A} = A \cup \partial A$ est appelée \emph{adhérence} de
  $A$.
\end{defn}
\begin{question}
  Dessiner la définition précédente.
\end{question}
\begin{defn}
  Soit $A$ une partie de $\R^n$ et $x \in A$. Un \emph{hyperplan
    d'appui} à $A$ en $x$ est un hyperplan $H$
  \begin{itemize}
  \item passant par $x$
  \item defini par un vecteur normal $\nu$ tel que
    $\forall y \in A, \langle y-x, \nu \rangle \leq 0$.
  \end{itemize}
\end{defn}
Un hyperplan d'appui à $A$ cantonne donc la partie $A$ à l'un des
demi-espaces qu'il definit.
\begin{question}
  Dessiner un exemple de partie
  \begin{itemize}
  \item n'ayant pas d'hyperplan d'appui en un point donné de son bord ;
  \item n'ayant aucun hyperplan d'appui ;
  \item ayant plus d'un hyperplan d'appui en un même point.
  \end{itemize}
\end{question}
\begin{prop}
  Une partie convexe de $\R^n$ admet un hyperlan d'appui en tout point
  de son bord\footnote{Celui-ci pouvant être vide.}.
\end{prop}
Cette propriété est une conséquence du théorème de séparations des
convexes fermés par des hyperplans.
\begin{question}
  On reprend les notations de la question (\ref{q:courbeniveau})
  \begin{enumerate}
  \item Donner un hyperplan d'appui à $\mc{C}_{\leq 4}(h)$.
  \item Justifier le fait que $\mc{C}_{\leq 2}(g)$ n'a pas d'hyperplan
    d'appui.
  \end{enumerate}
\end{question}

\subsubsection{Fonctions convexes}
\label{sec:fonctionsconvexe}

\begin{defn}
  Une fonction $f : \R^n \to \R$ est \emph{convexe} si
  \begin{itemize}
  \item son domaine de définition est une partie convexe de $\R^n$ ;
  \item $\forall x$, $y$ dans le domaine de définition de $f$,
    \[
      \forall t \in [0, 1], f(ty + (1-t)x) \leq tf(y) + (1-t)f(x).
    \]
  \end{itemize}
\end{defn}
\begin{question}
  Dessiner la définition précédente et en donner une interprétation
  géométrique.
\end{question}
\begin{defn}
  Une fonction $f$ est dite \emph{concave} si son opposé $-f$ est
  convexe.
\end{defn}
\begin{question}
  Donner des exemples de fonctions convexes.
\end{question}
\begin{question}
  Prouver, en n'utilisant que la définition, que la fonction
  $x \mapsto x^2$ définie sur $\R$ est convexe.
\end{question}
\begin{question}
  Justifier le fait
  \begin{enumerate}
  \item que la somme pondérée, de poids positifs de fonctions convexes est convexe ;
  \item qu'une fonctoin sous-linéaire est convexe? en donner des exemples ;
  \item que le $\max$ de fonctions convexes est convexe ;
  \item que la composition d'une fonction affine avec une fonction
    convexe est convexe.
  \end{enumerate}
\end{question}
\begin{question}
  Généraliser l'étude du $\max$ au contexte qui suit: soit
  $g : \R^n \times \R^p \times \R$ une fonction qui pour tout
  $x \in A \subset \R^n$ a une fonction partielle $y \mapsto g(x, y)$
  qui est convexe, montrer que $f(y) = \sup_{x \in A} g(x, y)$ est
  également convexe.
\end{question}
\begin{question}
  Montrer que les sous-niveaux d'une fonction convexe sont
  convexes. La réciproque est-elle vraie?
\end{question}


\section{Programmes linéaires dans le plan}
\label{sec:proglinplan}

Un programme linéaire est un problème d'optimisation qui prend la
forme
\[
\begin{PbOptim}{
    $\min$
  }{
    $f_0(x)$
  }{
    $f_i(x) \leq 0$, $1 \leq i \leq p$.
  }
\end{PbOptim}
\]
où la fonction objectif $f_0$ ainsi que les $f_i$ sont à valeurs
réelles et de domaine $\R^n$. Les programmes linéaires forment la plus
simple famille de problèmes d'optimisation contraints. Ils modélisent
cela dit la majeure partie des problèmes de recherche opérationnelle
classiques, ils permettent de résoudre les problèmes de plus courts
chemins dans un graphes, de flots maximaux ou encore de problèmes de
transports. Leur aspect unificateurs permet d'évaluer la difficulté de
résoudre ces problèmes les uns par rapport aux autres.

Le travail effectué jusqu'à présent pose déjà le cadre pour résoudre
autrement que par l'algorithme du simplexe ces types de problèmes. On
se limite au cas de la dimension $2$ afin de représenter les objets
qu'on manipule.

\begin{question}
  On revient sur les lieux géométriques décrits question
  (\ref{q:exmpdetravail}). On note $\mc{A}_u$ le lieu décrit par les
  contraintes
  \[
    \mc{A}_u : \begin{matrix}
      -x + 2 y & \leq & -1 \\
      x + y & \leq & 1
    \end{matrix}
  \]
  et $\mc{A}_b$ celui décrit par les contraintes précédentes
  auxquelles on ajoute
  \[
    x - 3y \leq 6.
  \]
  \begin{enumerate}
  \item Étudier le programme linéaire de lieu admissible $\mc{A}_u$ et
    mininisant $y$. Que se passe-t-il si l'on remplace $y$ par $-y$?
  \item A-t-on toujours une valeur minimal pour un programme linéaire
    ayant pour lieu admissible $\mc{A}_b$?
  \item Étudier le prgramme linéaire de lieu admissible $\mc{A}_b$ et
    minimisant $x + y$. Effectuer cette même étude pour la fonction
    objectif $-x-y$.
  \item Décrire une procédure géométrique qui permet de résoudre par
    acoups les programmes linéaires en dimension $2$.
  \end{enumerate}
\end{question}

\begin{question}
  Donner un exemple d'un programme linéaire:
  \begin{enumerate}
  \item non borné ;
  \item de lieu admissible non-borné mais de solution fini ;
  \item ayant une infinité de solution ;
  \item ayant une unique solution.
  \end{enumerate}
\end{question}

\begin{question}
  On considère le programme linéaire (\emph{P1}) suivant :
  \begin{displaymath}
    \begin{linearProg} {
        minimiser
      }{
        $f_0(x_1, x_2) = 3x_1 + 2x_2$
      }{
        \systeme{
          x_1 - x_2 \leq 0,
          4x_1 - x_2 \leq 1,
          -x_1 - x_2 \leq -5
        }
      }
    \end{linearProg}
  \end{displaymath}
  \begin{enumerate}
  \item Représenter le lieu admissible de (\emph{P1}) dans le plan
    euclidien.
  \item
    \begin{enumerate}
    \item[a.]  Tracer la courbe de niveau $6$ de la fonction objectif
      de (\emph{P1}). Elle sera notée $C_6$.
    \item[b.]  Indiquer les demi-espaces positif et négatif défini par
      $C_6$. Dans quelle direction doit-on translater $C_6$ afin de
      minimiser $f_0$.
    \end{enumerate}
  \item
    Tracer la courbe de niveau qui réalise le minimum de (\emph{P1})
    et calculer l'unique point optimal de (\emph{P1}). Quelle est la
    valeur optimale de (\emph{P1})?
  \item Résumer votre démarche en rappelant la condition d'optimalité
    d'une solution de (\emph{P1}).
  \end{enumerate}
\end{question}

\begin{question}
  On considère le programme linéaire (\emph{P2}) donné par
  \[
    \begin{linearProg} {
        minimiser
      }{
        $f_0(x_1, x_2) = x_1 + 2x_2$
      }{
        \systeme{
          x_2 \leq 1,
          x_1 \leq 1,
          -x_2 \leq 1,
          -x_1 \leq 1
        }
      }
    \end{linearProg}
  \]
  Résoudre (\emph{P2}) en suivant la démarche précédente.
\end{question}

\begin{question}
  On considère le programme linéaire (\emph{P3}) suivant :
  \[
    \begin{linearProg} {
        minimiser
      }{
        $x + 2y$
      }{
        \systeme{
          x + y \leq 1,
          -x + 2y \leq 2,
          x - 3y \leq 3
        }
      }
    \end{linearProg}
  \]
  \begin{enumerate}
  \item Représenter le lieu admissible de (\emph{P3}) dans $\R^2$.
  \item
    \begin{enumerate}
    \item[a.]  Tracer la courbe de niveau $0$ de la fonction objectif
      de (\emph{P3}). Elle sera notée $\mc{C}_0$.
    \item[b.]  Indiquer les demi-espaces positif et négatif définis
      par $\mc{C}_0$.
    \item[c.]  Indiquer dans quelle direction on doit translater
      $\mc{C}_0$ afin de minimiser la fonction objectif.
    \end{enumerate}
  \item Tracer la courbe de niveau qui réalise le minimum de
    (\emph{PL3}) et calculer l'unique point optimal de
    (\emph{P3}). Quelle est la valeur optimale de (\emph{P3})?
  \end{enumerate}
\end{question}

La démarche qui a été introduite précédemment permet de résoudre une
famille un peu plus large de problèmes d'optimisation dans $\R^2$.

\begin{question}
On considère la fonction différentiable $f : \R^2 \to \R$ donnée par
\[
f(x, y) = 3x^2 + y^2.
\]
\begin{enumerate}
\item Représenter les courbes de niveaux $2$ et $4$ de $f$ dans le
  plan euclidien.
\item À quel lieu correspond la condition $f(x, y) \leq 4$ ?
\item On s'intéresse au problème d'optimisation (\emph{P4})
  \[
  \begin{PbOptim}{
      minimiser
    }{
      $2x + y$
    }{
      $3x^2 + y^2 \leq 4$
    }
  \end{PbOptim}
  \]
  Représenter la courbe de niveau de la fonction objectif qui
  correspond à la valeur optimale de (\emph{P4}).
\item Comment trouver le point optimale correspondant à (\emph{P4})?
  Faire le calcul.
\end{enumerate}
\end{question}

\begin{question}
  On considère le problème d'optimisation (\emph{P5}) suivant
  \[
    \begin{PbOptim}{
        minimiser
      }{
        $x + y$
      }{
        $\begin{matrix}
          x+2y \leq 3 \\
          x \in B
        \end{matrix}$
      }
    \end{PbOptim}
  \]
  où $B$ est la partie de $\R^2$ qu'on a décrit section
  (\ref{q:partieB}).
  \begin{enumerate}
  \item Dessiner le lieu admissible du problème de (\emph{P5}).
  \item Représenter la courbe de niveau de la fonction objective qui
    réalise le minimum de (\emph{P5}).
  \item Calculer le point optimal ainsi que la valeur optimale de
    (\emph{P5}).
  \end{enumerate}
\end{question}

La résolution de problèmes d'optimisation à la main comme on vient de
le faire se heurte à deux obstacles:
\begin{itemize}
\item les dimensions des espaces concernés, qui représentent souvents
  des \textit{features} d'un dataset, dépassent de loin les dimensions
  représentables graphiquement ;
\item les fonctions concernées, que cela soit au niveau des
  contraintes ou des fonctions objectifs ne sont pas souvent
  linéaires.
\end{itemize}
Cela étant dit la démarche qu'on vient de mettre en évidence ici est
le point de départ de l'essentiel des démarches itératives pour
résoudre les problèmes d'optimisation. Pour être en mesure d'aborder
cette généralisation il nous faudra, pour commencer, étudier les
conditions sous lesquelles on est en mesure d'approcher localement et
par des fonctions affines les courbes de niveau d'une fonction
objectif. Cette étude recoupe le besoin de généraliser les notions de
dérivabilité et dérivée d'une fonction numérique de domaine réel au
cas multivarié.

\appendix

\section{Pivot de Gauss}
\label{sec:pivotGauss}

On se donne une matrice $M \in \mc{M}_{m, n}(\R)$. Une opération
élémentaire sur $M$ est une des deux opérations suivantes:
\begin{itemize}
\item
  pour $i \in \{1, \ldots, m\}$ et $j \in \{1, \ldots, n\}$
  interchanger les lignes $L_i$ et $L_j$
  \[
  L_i \leftrightarrow L_j
  \]
\item
  pour $i \in \{1, \ldots, m\}$, $j \in \{1, \ldots, n\}$, $i \neq j$,
  $\lambda_i \in \R^*$ et $\lambda_i \in \R$, remplacer la ligne $L_i$
  par $\lambda_i L_i + \lambda_jL_j$ :
  \[
  L_i \leftarrow \lambda_iL_i + \lambda_jL_j.
  \]
\end{itemize}
Le pivot de Gauss est un algorithme de transformation d'une matrice
$M$ en matrice triangulaire supérieure et éventuellement en matrice
identité, dans le but d'inverser une matrice, de calculer un
déterminant ou de résoudre un système linéaire. Il est d'utilisation
constante en calcul numérique.\footnote{L'expression de l'inverse
  d'une matrice inversible à l'aide des cofacteurs est à prohiber!
  Elle est théoriquement intéressante et particulièrement esthétique
  mais numériquement inéfficace.}

\subsection{Opérations élémentaires}
\label{subsec:opElementaires}

On traduit par la suite les opérations élémentaires précédentes à
l'aide du produit matriciel.
\begin{question}
  Trouver des matrices $P(i, j)$ et $U(i, j, \lambda, \nu)$ dont la
  multiplications avec $M$ réalise les opérations élémentaires
  précédentes.
\end{question}

\begin{question}
  Trouver les matrices $P_t(i, j)$ et $U_t(i, j, \lambda, \nu)$ qui
  réalisent les opérations précédentes sur les colonnes.
\end{question}

\subsection{Systèmes linéaires}

\begin{question}
  Déduire de la section \ref{subsec:opElementaires} que les opérations
  élémentaires du pivot de Gausse transforment tout système linéaire
  dont $M$ est une matrice (si appliqué des deux côtés de l'égalité)
  en un système linéaire ayant les mêmes solutions.
\end{question}

\begin{question}
  Comment procéder pour décrire toute les solutions d'un système
  linéaire?
\end{question}

\begin{question}
  Discuter de l'efficacité de l'implémentation de l'algorithme
\end{question}

\subsection{Déterminant}

\begin{question}
  Retrouver les relations standards sur le déterminant d'une matrice
  qui subit une opération élémentaire à l'aide de la formule
  \[
  \det(AB) = \det(A)\det(B).
  \]
\end{question}

\begin{question}
  Comment procéder pour calculer le déterminant d'une matrice à l'aide
  du pivot de Gauss?
\end{question}

\subsection{Matrices équivalentes}

\noindent Deux matrices $A$ et $B$ sont dites équivalentes s'il
existe deux matrices inversibles $P$ et $Q$ telle que
\[
A = PBQ.
\]
\begin{question}
  Justifier le fait que toute matrice est équivalente à une matrice
  ayant un bloc identité et des zéros partout ailleurs.
\end{question}


\section{Géométrie affine}

Cette feuille est centrée autour des applications affines de $\R^n$,
plus particulièrement autours des projections et transformations de
l'espace affine euclidien. L'objectif est de vous apporter les
éléments de langage et les résultats de structure qui vous permettent
de les expliciter et les utiliser dans les contextes de ML. Pour
rappel la majeure partie des algorithmes de ML sont de nature
géométrique.

\subsection{Applications affines}
\label{sec:defnappaffine}

Cette section est un kit de survie en milieu hostile. Il s'agit de
s'armer des différentes notions de transformations et applications
affines dont vous aurez l'usage lors de vos cours de ML. Notre
démarche dans la suite est particulièrement pragmatique\footnote{Elle
  pourrait heurter certaines âmes sensibles et quelques matheux qui ont
  gardés un peu de sens esthétique.}; on se limite au cas des
applications affines de $\R^n$ dans $\R^m$.
\begin{defn}
  Une application $f : \R^n \to \R^m$ est dite affine s'il existe un
  point $\bs{a} \in \R^m$ tel que $f - \bs{a}$ soit une application
  linéaire.
\end{defn}
\noindent Cette définition formelle nécessite une réalisation plus concrète afin
qu'on puisse l'utiliser. Pour cela on va introduire une généralisation
de la notion de bases dans le cas de $\R^n$ qui nous permettra de se
placer en un point quelconque pour y travailler.
\begin{defn}
  Un repère de l'espace affine $\R^n$ est la donnée d'un point
  $\bs{o}$ appelé origine du repère et d'une base
  $\bs{v} = (v_1, \ldots, v_n)$ de $\R^n$. On note
  $(\bs{o}, \bs{v})$ une telle donnée.
\end{defn}
\noindent Le repère canonique de $\R^n$ est le repère
$(\underline{0}, e_1, \ldots, e_n)$, que vous aviez eu l'habitude de
manipuler dans le secondaire et en physique. Du moins dans les cas de
dimensions $1$ ou $2$.
\begin{defn}
  Soient $(\bs{o}, \bs{v})$ et $(\bs{l}, \bs{w})$ deux repères
  respetivement de $\R^n$ et $\R^m$. L'écriture de $f$ dans les
  repères $(\bs{o}, \bs{v})$ et $(\bs{l}, \bs{w})$ est la donnée d'une
  matrice $M$ telle que pour tout $x \in \R^n$,
  \begin{equation}
    \label{defn:affine}
    f(x) = M(x - \bs{o})_{\bs{v}} + f(\bs{o}).
  \end{equation}
  où $\bullet_{\bs{v}}$ désigne l'écriture du vecteur $\bullet$ dans
  la base $\bs{v}$. Le membre de droite de l'équation
  \ref{defn:affine} est sous-entendu être décrit dans le repère
  $(\bs{l}, \bs{w})$. Cette équation est parfois appelée équation de
  $f$ dans les repères $(\bs{o}, \bs{v})$ et $(\bs{l}, \bs{w})$.
\end{defn}
\begin{exmp}
  Une application affine peut être donné par une experssion dans le
  repère canonique
  \[
  x \mapsto Mx + b
  \]
  où $x$ et $b$ sont des vecteurs respectivement dans $\R^n$ et $\R^m$
  et $M \in \mc{M}_{m, n}(\R)$. Par exemple
  \[
  \begin{pmatrix} x \\ y \end{pmatrix} \mapsto \begin{pmatrix} 1 & 1
    \\ 2 & -1 \end{pmatrix}\begin{pmatrix} x \\ y \end{pmatrix}
  + \begin{pmatrix} 3 \\ -2 \end{pmatrix}
  \]
  Ce qui nous donne une fonction de $\R^2$ dans $\R^2$ affine en $x$
  et $y$ au sens qu'on entend intuitivement.
\end{exmp}
Dans le but de trouver des expressions agréables ou standardisées
d'une application affine on est amené à écrire ces applications
affines dans des repères adaptés. Pour pouvoir exprimer un tel
contexte il nous faut être en mesure de formaliser la notion
de changement de repères.
\begin{defn}
  Le changement de repères de $\R^n$ de $(\bs{o}, \bs{v})$ vers
  $(\bs{l}, \bs{w})$ est l'application affine donnée dans les repères
  précédents par
  \[
  x \longmapsto P_{\bs{v}}^{\bs{w}}(x - \bs{o})_{\bs{v}} + \bs{l}
  \]
  où $P_{\bs{v}}^{\bs{w}}$ est la matrice de passage de la base
  $\bs{v}$ vers la base $\bs{w}$ ; c'est-à-dire la matrice des
  vecteurs de $\bs{v}$ écrits dans la base $\bs{w}$.
\end{defn}
\noindent Voici enfin comment formaliser le passage de l'écriture
d'une application dans un repère vers un autre.
\begin{prop}
  Soient $(\bs{o}, \bs{v})$ et $(\bs{o'}, \bs{v'})$ deux bases de
  $\R^n$, $(\bs{l}, \bs{w})$ et $(\bs{l'}, \bs{w'})$ deux bases de
  $\R^m$. On note $M$ la matrice d'une application affine $f$ dans les
  repères $(\bs{o}, \bs{v})$ et $(\bs{l}, \bs{w})$, et $M'$ celle de
  $f$ dans les repères $(\bs{o'}, \bs{v'})$ et $(\bs{l'}, \bs{w'})$.
  Si $P_{\bs{v'}}^{\bs{v}}$ désigne la matrice de passage de $\bs{v}$
  vers $\bs{v'}$ et $P_{\bs{w'}}^{\bs{w}}$ la matrice correspondante
  dans le cas de $\bs{w}$ et $\bs{w'}$ alors pour tout $x \in \R^n$
  \[
  M'(x - \bs{o'})_{\bs{v'}} + f(\bs{o'}) = f(x) =
  P_{\bs{w'}}^{\bs{w}}M\left(P_{\bs{v'}}^{\bs{v}}\right)^{-1}(x -
  \bs{o})_{\bs{v}} + f(\bs{o})
  \]
\end{prop}
\begin{question}
  Écrire la fonction du premier exemple dans le repère de $\R^2$ donné
  au départ et à l'arrivée par
  $\big((0, 0), \{(1, 1), (1, -1)\}\big)$.
\end{question}
\begin{rem}
  Dans les faits, on ne fera pas souvent ce type de changement de
  repères. À partir des propriétés des applications à l'étude on sera
  à même de trouver un bon repère dans lesquels les décrire. Il reste
  que de tels changements de repères doivent être compris et faits
  quand nécessaire.
\end{rem}
\begin{exmp}
  Les applications affines constantes sont celles dont l'expression
  dans tous les repères a une matrice qui est nulle.
\end{exmp}
\begin{exmp}
  Avec la définition qu'on a donné toute application linéaire va
  décrire une application affine.
\end{exmp}

\subsubsection{Translations et homothéties}

On s'attarde un instant sur les plus simples des applications affines
non triviales.
\begin{exmp}
  La translation de vecteur $b$ est l'application affine donnée dans
  le repère canonique de $\R^n$ par $x \mapsto x + b$.
\end{exmp}
\begin{exmp}
  L'\emph{homothétie} de rapport $\lambda$ et centre $\bs{o}$ est
  l'application affine de $\R^n$ dans lui-même dont l'écriture dans le
  repère $(\bs{o}, \bs{v})$ prend la forme
  \[
  x \mapsto \lambda (x - \bs{o})_{\bs{v}} + \bs{o}
  \]
  L'homothétie de rapport $1$ est simplement l'identité.
\end{exmp}
\begin{question}
  Donner une particularité que partagent les homothéties et les
  translations.
\end{question}
\begin{question}
  Que donne la composition de deux translations? Celle de deux
  homothéties?
\end{question}
\begin{question}
  Dessiner l'image du carré de sommets $(\pm 1, \pm 1)$
  dans $\R^2$ par les homothéties suivantes:
  \begin{itemize}
  \item
    de rapport $2$ et de centre l'origine
  \item
    de rapport $2$ et de centre $(1, 0)$.
  \end{itemize}
\end{question}

\subsubsection{Projections affines}

Les projections affines sont la généralisation des projections
linéaires. Pour rappel une projection $p : E \rightarrow E$ de
l'espace vectoriel $E$ sur lui même est un application linéaire qui
satisfait la relation $p^2 = p$. Cette relation garanti le fait que
$G = \Ker(p)$ et $F = \Ker(p-\id)$ sont des espaces vectoriels
supplémentaires dans $E$. Ainsi, pour tout $x = x_F + x_G$, la
projection $p$ est définie par $p(x) = x_F$. L'application $p$ est
appelée projection sur $F$ parallèlement à $G$.
\begin{defn}
  Une application affine $f$ de $\R^n$ dans $\R^n$ est une projection
  affine si sa matrice dans un repère est la matrice d'une projection
  linéaire.
\end{defn}
\noindent Une projection affine, dans un repère $(\bs{o}, \bs{v})$ de
$\R^n$, s'écrit donc sous la forme
\[
x \mapsto M(x - \bs{o})_{\bs{v}} + \bs{o}
\]
On voit dans cette écriture que $\bs{o}$ est fixé par la
projection. On vient d'écrire la projection de $\bs{o} + \Ker(M-I_n)$
parallèlement à $\bs{o} + \Ker(M)$.
\begin{question}
  Étudier des exemples de projections affines dans $\R^2$. Dessiner
  systématiquement les droites affines qui caractérisent cette
  projection.
\end{question}

\subsection{Transformations orthogonales}

\subsubsection{Aparté sur l'orthogonalité}

On se donne un produit scalaire $\langle \cdot, \cdot \rangle$ sur
$\R^n$\footnote{Pensez au produit scalaire usuel si \c{c}a vous
  facilite la vie.}. On rappelle que c'est une forme bilinéaire
symétrique définie positive. Elle vient avec une norme associée dite
norme euclidienne définie par
\[
\|x\| = \sqrt{<x, x>}.
\]
\begin{defn}
  Deux vecteurs $x$, $y$ in $\R^n$ sont dit orthogonaux pour
  $\langle \cdot, \cdot \rangle$ si $\langle x, y \rangle = 0$.
\end{defn}
\noindent Nous avons vu que cette notion correspond dans le cas du
produit scalaire usuel au fait que l'angle entre $x$ et $y$ est
$\pi/2$ $[\pi]$.
\begin{defn}
  Une base $\bs{v}$ est dite \emph{orthognale} si les vecteurs qui la
  constituent sont orthogonaux deux à deux. Elle est
  \emph{orthonormale} si ses vecteurs sont de norme $1$.
\end{defn}
\noindent L'exemple le plus simple d'une base orthonormée est celui de
la base canonique de $\R^n$ pour le produit scalaire usuel.
\begin{question}
  Donner d'autres bases orthonormales de $\R^3$.
\end{question}
\begin{question}
  Comment construire une base orthonormée de $\R^3$ à partir de la
  base
  \[
  \{(1, 0, 1), (0, 1, 1), (1, 1, 0)\}?
  \]
\end{question}
\begin{defn}
  Étant donné une partie $K$ de $\R^n$ on appelle orthogonal de $K$
  pour $\langle \cdot, \cdot \rangle$ le sous-espace vectoriel
  \[
  K^\perp = \{ x \in \R^n \mid \forall y \in K, \; \langle x, y \rangle = 0 \}.
  \]
\end{defn}
\noindent Ainsi un hyperplan $H$ de $\R^n$ quand il est donné implicitement est
décrit comme l'orthogonal d'un vecteur de $\R^n$ qu'on appelle un
vecteur \emph{normal}.
\begin{question}
  Décrire dans $\R^3$ l'orthogonal à la partie $\{(1, 1, 1), (1, 0, 1)\}$.
\end{question}

\subsubsection{Transformations vectorielles}

La notion de produit scalaire permet à la fois de formaliser la notion
d'angle ainsi que celle de distance\footnote{On étudiera plus en
  détail la seconde question un peu plus tard.}. Les transformations
affines de $\R^n$, c'est-à-dire les applications affines bijectives,
qui présèrvent les angles et les distances sont d'un grand intérêt
géométrique; elles ne touchent pas aux dispositions relatives de
configurations de parties dans $\R^n$ et donc présèrvent les
propriétés géométriques de celles-ci. Elles présèrvent par exemple la
perpendicularité ou le parallélisme. Pour l'instant on se limite au
cas vectoriel avant d'exposer la situation affine, plus générale.

Avant de poursuivre notre travail on prend un instant pour étudier de
plus prêt une écriture matricielle de $\langle \cdot, \cdot \rangle$.
Étant donné une base $\bs{v}$ de $\R^n$, on appelle matrice de
$\langle \cdot, \cdot \rangle$ dans la base $\bs{v}$ la matrice
\[
\bs{A} = \big(\langle v_i, v_j \rangle\big)_{i, j}.
\]
\begin{question}
  Montrer que, dans la base $v$, on a pour tout $x$, $y \in \R^n$,
  $\langle x, y \rangle = x^TAy$.
\end{question}
Si $\bs{v}$ est une base orthonormale dans ce cas $A = I$ et on a,
pour tout $x$, $ y \in \R^n$ écrits dans cette base
\[
\langle x, y \rangle = x^Ty.
\]
Ce qui nous ramène à l'écriture usuelle du produit scalaire.
\begin{hyp}
  Quitte à se ramener à une base orthonormée
  appropriée\footnote{\c{C}a existe toujours. Voir algorithme de
    Gram-Schmidt.} on peut supposer que notre produit scalaire a la
  forme usuelle.
\end{hyp}
Soit $M$ une matrice inversible dans une base orthonormée de
$\R^n$. Dire que $M$ préserve le produit scalaire
$\langle \cdot, \cdot \rangle$ signifie que pour tout $x$,
$y \in \R^n$ on a
\[
(Mx)^T(My) = x^Ty.
\]
D'où l'on obtient, pour tout $x$, $y \in \R^n$
\[
x^T\left(M^TM - I_n\right)y = 0.
\]
En prenant $x$ et $y$ ayant des coefficients nuls sauf un on peut
montrer que les coefficients de $M^TM - I_n$ sont tous nuls. Donc
\[
M^TM = I_n.
\]
\begin{prop}
  Une isomorphisme de $\R^n$ préserve un produit scalaire si et
  seulement sa matrice dans une base orthonormée satisfait
  \[
  M^TM = I_n.
  \]
  Une matrice qui satisfait la propriété précédente est dite
  \emph{orthogonale}.
\end{prop}
\begin{question}
  Quel peut être le déterminant d'une matrice orthogonale?
\end{question}
\begin{question}
  Traduire la proposition précédente en fonction des vecteurs colonnes
  de la matrice $M$.
\end{question}
\noindent On étudie un premier exemple de matrices orthogonales. On se
fixe une base orthonormée de $\R^n$ dans laquelle on travaille. Soit
$\theta \in \R$ on note $R(\theta)$ la matrice
\[
R(\theta) =
\begin{pmatrix}
  \cos(\theta) & - \sin(\theta) \\
  \sin(\theta) & \cos(\theta)
\end{pmatrix}.
\]
\begin{question}
  Montrer que $R(\theta)$ est une matrice orthogonale pour tout
  $\theta$. Quel est son déterminant?
\end{question}
\begin{question}
  Dessiner l'image du carré de points extrémaux $(\pm 1, \pm 1)$ par
  $R(\theta)$ pour $\theta = 0$, $\pi/4$.
\end{question}
Un exemple de matrice orthogonale qui n'est pas de déterminant $1$ est
celui-ci
\[
s =
\begin{pmatrix}
1 & 0 \\
0 & -1
\end{pmatrix}.
\]
\begin{question}
  Interpréter géométriquement cette matrice.
\end{question}
\begin{question}
  Dessiner l'image du carré de points extrémaux
  $\{(0, 0), (0, 1), (1, 0), (1, 1)\}$
  par $R(\theta)$ pour $\theta = 0$, $\pi/4$.
\end{question}
\begin{question}
  Déterminer toute les matrices orthogonales de $\R^2$ muni de son
  produit scalaire usuel. Vous pourrez les étudier suivant leurs
  spectres.
\end{question}
\begin{question}
  Donner des exemples de matrices orthogonales en toute
  dimension. Essayez d'être le plus général possible.
\end{question}
\begin{question}
  Étendre la classification des matrices orthogonales au cas de
  dimension $3$.
\end{question}

\subsubsection{Transformations affines}

\emph{TODO}

\section{Les formes bilinéaires symétriques}
\label{sec:FBS}

Les formes bilinéaires symétriques apparaissent sous différents
aspects en mathématiques parmi ceux-ci, en réalité dès que l'on
travaille avec une matrice symétrique. On peut par exemple évoquer le
produit scalaire sur $\R^n$ qui nous permet de décrire un objet qui
donne à la fois une information sur la norme d'un vecteur et l'angle
que deux vecteurs forment. On peut encore évoquer le cas de la matrice
de covariance, d'usage fréquent et symétrique. Ou enfin les polynôme
de degré $2$ ayant un nombre quelconque de variables.

\subsection{Définition}
\label{defn:formequadratique}

\begin{defn}
  Soit $E$ un $\R$-espace vectoriel. Une \emph{forme bilinéaire} est
  une application $\phi : E \times E \to \R$ pour laquelle les
  applications partielles en tout point $x \in E$
  \begin{itemize}
  \item $t \mapsto \phi(x, t)$
  \item $t \mapsto \phi(t, x)$
  \end{itemize}
  sont linéaires.
\end{defn}
\begin{exmp}
  La multiplication matricielle
  $\times : \mc{M}_n(\R) \times \mc{M}_n(\R) \to \R$ est une
  application bilinéaire sur l'espace vectoriel des matrices carrées
  de taille $n$.
\end{exmp}
\begin{defn}
  Une forme bilinéaire $\phi$ sur un $\R$-espace vectoriel $E$ est dite
  \begin{itemize}
  \item \emph{symétrique} si $\forall x$, $y \in E$, $\phi(x, y) = \phi(y, x)$ ;
  \item \emph{antysmétrique} si $\forall x$, $y \in E$,
    $\phi(x, y) = - \phi(y, x)$ ;
  \item \emph{positive} si $\forall x$, $\phi(x) \geq 0$ ;
  \item \emph{négative} si $\forall x$, $phi(x) \leq 0$ ;
  \item \emph{définie} si $\forall x$, $\phi(x) = 0 \Rightarrow x = 0$.
  \end{itemize}
\end{defn}
\begin{question}
  Justifier le fait que le produit scalaire usuel sur $\R^n$ est une
  forme bilinéaire symétrique définie positive.
\end{question}
Dans la pratique, et quand on travaille sur $\R^n$, une forme
bilinéaire symétrique est donnée par une matrice symétrique $P$ de
taille $(n, n)$. L'application bilinéaire $\Phi$ associée s'écrit
\[
\forall x, y \in \R^n, \quad \Phi(x, y) = x^TPy.
\]
L'écriture à droite de l'égalité donne effectivement une forme
bilinéaire symétrique. Si l'on s'est placés dans la base
$b_1, \ldots, b_n$ pour écrire les vecteurs $x$ et $y$, la matrice $P$
est la matrice de coefficients $P_{ij} = b_i^TPb_j$.

\subsection{Les formes quadratiques}
\label{sec:fromquadra}

Les formes bilinéaires symétriques donnent naissance aux formes
quadratiques. Celles-ci donnent des programmes d'optimisations qu'on
résout assez bien par les méthodes introduites dans ce cours. On passe
d'un contexte linéaire dans le cas des programmes linéaires à celui de
degré $2$.
\begin{defn}
  Soit $E$ un espace vectoriel sur $\R$. On dit que $Q : E \to \R$ est une
  forme quadratique sur $E$ s'il existe une forme bilinéaire
  symétrique $\phi: E\times E \to \R$ telle que pour tout $x \in E$,
  $Q(x) = \phi(x, x)$.
\end{defn}
\noindent Cette définition permet déjà de voir qu'étant donné un
produit scalaire $\langle \cdot\, ,\cdot\rangle$ sur $E$,
l'application $x \mapsto \langle x, x \rangle$ est une forme
quadratique ; c'est la norme au carré (la norme associée au produit
scalaire).  Une forme quadratique $Q$ sur $\R^n$ est donc la donnée
d'une matrice symétrique $P$ telle que
\[
\forall x \in \R^n, \quad Q(x) = x^TPx.
\]
\begin{exmp}
  La matrice carrée $P = \begin{pmatrix} 1 & 2 \\ 2 & 0 \end{pmatrix}$
  donne lieu à la forme quadratique $Q$ l'expression
  pour $(x, y) \in \R^2$ est
  \[
  Q\big((x, y)\big) = x^2 + 4xy.
  \]
\end{exmp}
\noindent On remarque que l'expression obtenue est un polynôme de degré
total\footnote{C'est le degré du polynôme obtenu en rempla\c{c}ant
  toutes les variables par une unique variable $t$.} $2$. Ce fait est
général : étant donné une forme quadratique non nulle $Q$ sur $\R^n$,
l'expression $Q(x)$ est un polynôme de degré total $2$ en les
coordonnées du vecteur $x \in \R^n$.
\begin{rem}
  Par abus, dans le cas de $\R^n$, ces adjectifs sont accolés à la
  matrice symétrique $P$ telle que $Q(x) = x^TPx$.
\end{rem}

\begin{question}
  Donner, quand possible, un exemple de forme quadratique sur $\R^2$ qui
  soit (n'oubliez pas de justifier vos réponses):
  \begin{itemize}
  \item
    positive, non définie ;
  \item
    définie positive;
  \item
    définie non positive ni négative.
  \end{itemize}
  Dessiner un croquis illustrant le graphe de la fonction quadratique
  donnée dans chacun des cas\footnote{Vous êtes autorisés
    (encouragés?) à utiliser \texttt{matplotlib}.}.
\end{question}
\begin{solution}
  Voici quelques exemples de formes quadratiques répondant à la
  question précédente :
  \begin{itemize}
  \item On cherche une forme quadratique positive, non
    définie, par exemple la forme quadratique $Q$ sur $\R^2$ donnée
    par $Q\big((x,y)\big) = x^2$. Elle est évidemment positive, pour
    voir qu'elle est non définie il suffit de remarquer que
    $Q\big((0,1)\big) = 0$. Il existe donc un vecteur non nul dont
    l'image par $Q$ est nulle, l'implication
    $Q(X) = 0 \Rightarrow X = \underline{0}$ n'est pas satisfaite.
  \item Pour une forme quadratique positive définie tout
    simplement la forme quadratique associée au produit scalaire
    usuel.
  \item Cette question est piège, il n'existe pas de forme
    quadratique définie non positive et non négative. En effet, si une
    telle forme quadratique $Q$ existe sur $\R^n$ alors il existe deux
    vecteurs de $\R^n$, $X_0$ et $Y_0$ tels que $Q(X_0) < 0$ (définie
    non positive) et $Q(Y_0) > 0$ (définie non négative). Voici deux
    arguments qui montre l'impossibilité d'une telle
    situation\footnote{Vous pouvez en trouver d'autres, certainement.}
    :
    \begin{itemize}
    \item $Q$ est une application continue car polynomiale en les
      coordonnées usuelles de $\R^n$. Le segment de la droite affine
      $[X_0, Y_0]$ ne passe pas par l'origine\footnote{Y réfléchir
        deux secondes ; l'image d'une droite vectorielle par une forme
        quadratique ne change pas de signe.} son image est un connexe
      par arcs de $\R$ car il en est de même de $[X_0, Y_0]$. C'est
      donc un intervalle $I$ de $\R$. Cet intervalle contient une
      valeur strictement négative et une autre strictement positive,
      il contient donc $0$ et il existe donc un vecteur \emph{non nul}
      envoyé par $Q$ sur $0$.
    \item On note $\Phi$ la forme bilinéaire symétrique associée à
      $Q$. Soit $\lambda$, $\mu$ deux scalaires dans $R$. On considère
      l'expression en $\lambda$, $\mu$
      \[
      Q(\lambda X_0 + \mu Y_0) = \lambda^2Q(X_0) +
      2\lambda\mu\Phi(X_0, Y_0) + \mu^22Q(Y_0).
      \]
      Pour chaque $\mu \in \R$ c'est une expression quadratique en
      $\lambda$, le coefficient de $\lambda^2$ n'étant pas nul par
      hypothèse. L'équation
      \begin{equation}
        \label{eq:quadra}
        Q(\lambda X_0 + \mu Y_0) = 0
      \end{equation}
      est donné par
      \[
      \Delta = 4\mu^2\Phi(X_0, Y_0)^2 - 4\mu^2Q(X_0)Q(Y_0) \geq 0
      \]
      car le premier terme est positif et le second négatif. Pour
      $\nu \neq 0$ on a donc nécessairement une solution non nulle car
      le terme constant de l'équation \eqref{eq:quadra} en $\lambda$
      ne l'est pas.
    \end{itemize}
  \end{itemize}
\end{solution}
On admet par la suite le résultat important suivant:
\begin{thm}
  \label{thm:reducsym}
  Toute matrice symétrique $P \in M_n(\R)$ est diagonalisable sur
  $\R$ dans une base orthonormée.
\end{thm}
\noindent Ce résultat, important, est valable sous des conditions beaucoup plus
larges sur le corps de base ; on le  montre en partie à l'aide d'un
algorithme dit de Gauss.
\begin{question}
  \label{qu:critereposdefn}
  À l'aide du déterminant d'une matrice carrée, donner une condition
  sur une forme quadratique $Q$ sur $\R^n$ afin que celle-ci soit:
  \begin{itemize}
  \item
    définie ;
  \item
    non positive et non négative.
  \end{itemize}
  Pouvez-vous en dire plus dans le cas des formes quadratiques sur
  $\R^2$? Sait-on, dans ce cas, quand une forme quadratique est
  positive? Comment différencier le cas positif du cas négatif?
  Formaliser le cas des formes quadratiques sur $\R^2$.
\end{question}
\begin{solution}
  D'après le théorème \eqref{thm:reducsym} toute forme quadratique
  dans une base orthonormée bien choisie de vecteurs propres
  $(v_1, \ldots, v_n)$ prend la forme
  \[
  Q\Big(\sum_{i=1}^n x_i v_i\Big) = \sum_{i=1}^n \lambda_i x_i^2
  \]
  où $\lambda_i$ désigne la valeur propre associée à $v_i$. La matrice
  de la forme bilinéaire associée à $Q$ dans cette base prend la forme
  \[
  \begin{pmatrix}
    \lambda_1 & 0 & \hdots & 0 \\
    0 & \lambda_2 & \ddots & \vdots \\
    \vdots & \ddots & \ddots & 0 \\
    0 & \hdots & 0 & \lambda_n.
  \end{pmatrix}
  \]
  Il est clair désormais, qu'il faut et il suffit que la matrice
  associée à $Q$ ait toutes ses valeurs propres de même signe et non
  nulles pour que $Q$ soit définie\footnote{Si tel n'était pas le cas
    on se retrouverait dans le cas de l'exercice précédent.}. Pour que
  $Q$ ne soit ni positive, ni négative il faut et il suffit que cette
  même matrice ait au moins deux valeurs propres de signes opposés.

  Dans le cas des formes quadratiques sur $\R^2$ on n'a que deux
  valeurs propres. Pour s'assurer de la définition il suffit d'avoir
  un déterminant strictement positif. Dans ce cas les deux valeurs
  propres ne sont pas nulles et, de plus, de même signe. Pour vérifier
  la positivité sous une telle condition il suffit de regarder le
  signe de la trace de la matrice associée. \\
  Si le déterminant est strictement négatif les deux valeurs propres
  sont de signes distincts et on tombe dans le cas d'une forme
  quadratique ni positive, ni négative.
\end{solution}

\begin{question}
  Suffit-il qu'une matrice symétrique ait des coefficients positifs
  pour que la forme quadratique associée le soit?
\end{question}

\begin{solution}
  Non. On considère la forme quadratique $Q$ donnée par
  \[
  Q\big((x,y)\big) = x^2 + xy + y^2.
  \]
  En complétant le carré\footnote{Cette démarche est la première étape
    de l'algorithme de Gauss.} ne $x$ on obtient
  \[
  Q\big((x, y)\big) = \left(x + \frac{y}{2}\right)^2 - \frac{y^2}{4}
  \]
  En prenant $x = 1$ et $y = -2$ on obtient
  \[
  Q\big((1, -2)\big) = -1.
  \]
\end{solution}

\begin{question}
  \label{qu:convexiteP}
  Donner des conditions suffisantes sur la positivité de la matrice
  associée à une forme quadratique pour que celle-ci soit
  \begin{itemize}
  \item convexe ou concave ;
  \item ni l'un ni l'autre.
  \end{itemize}
  Ces conditions sont-elles nécessaires?
\end{question}

\begin{solution}
  La convexité de la fonction $Q$ donnée par $Q(x) = x^TPx$ correspond
  à la négativité, pour tout $\theta \in [0,1]$ et tout
  $x, y \in \R^n$, de l'expression
  \[
  E(x,y, \theta) = Q\big((1-\theta)x + \theta y\big) -
  \big((1-\theta)Q(x) + \theta Q(y)\big).
  \]
  On va faire en sorte de tirer de cette expression une condition
  utilisable.
  \begin{align}
    E(x, y, \theta) & = \big((1-\theta)x + \theta y\big)^TP\big((1-\theta)x + \theta y\big)
                      - \big((1-\theta)x^TPx + \theta y^TPy\big) \\
    \intertext{en développant le premier terme à gauche}
    E(x, y, \theta) & = \big((1-\theta)x\big)^TP\big((1-\theta)x + \theta y\big) +
      \big(\theta y)^TP\big((1-\theta)x + \theta y\big) \notag \\
    & \phantom{=}\quad  - \big((1-\theta)x^TPx + \theta y^TPy\big)\\
    \intertext{en regroupant intelligemment les termes\footnote{Yes, you can!}\footnote{Bug!}}
    E(x, y, \theta) & = \big((1-\theta)x\big)^TP\big((1-\theta)x + \theta y\big) - (1-\theta)x^TPx \notag \\
    & \phantom{=}\quad + \big(\theta y)^TP\big((1-\theta)x + \theta y\big) - \theta y^TPy\\
\intertext{la suite est automatique}
    E(x, y, \theta) & = \big((1-\theta)x\big)^TP\big((-\theta)x + \theta y\big) +
      \big(\theta y)^TP\big((1-\theta)x + (\theta-1)y\big) \\
    & = (1-\theta)\theta\left[x^TP(y-x) - y^TP(y-x)\right] \\
    & = -(1-\theta)\theta(y-x)^TP(y-x)\\
    & = -(1-\theta)\theta Q(y-x).
  \end{align}
  Ainsi $E(x, y, \theta)$ est négative si et seulement si $Q(y-x)$ est
  positive. Donc $Q$ est convexe si et seulement si pour tout $x$, $y$,
  $Q(y-x) \geq 0$. Il n'est pas difficile de voir que cette condition
  correspond au fait que $Q$ soit positive.

  J'ai voulu ici vous montrer qu'il est possible d'aller au bout de
  calculs qui peuvent vous sembler un peu horrible. La dernière
  condition qu'on obtient peut être obtenue autrement à l'aide de la
  condition d'ordre $1$ sur la convexité. Le calcul dans ce cas là est
  plus rapide.
\end{solution}


\subsection{Sous-ensembles des matrices symétriques}
\label{sec:symmatrices}

On note $S_n$ l'ensemble des matrices symétriques de
$M_n(\R)$. L'ensemble des matrices symétriques positives
(resp. définies positives) est noté $S_n^+$ (resp. $S_n^{++}$). Les
notations dans le cas négatif sont $S_n^{-}$ et $S_n^{--}$.

\begin{question}
  On s'intéresse à la structure de $S_n$.
  \begin{enumerate}
  \item Montrer que $S_n$ est un sous-espace vectoriel de
    $M_n(\R)$. Peut-on en dire plus? Calculer la dimension de $S_n$.
  \item Montrer que $S_n^{+}$ est une partie convexe de
    $S_n$. Peut-on en dire autant de $S_n^{++}$?
  \item Est-ce que le résultat précédent est valable dans le cas de
    $S_n^{-}$?
  \end{enumerate}
\end{question}

\begin{solution}
  \begin{enumerate}
  \item On montrer que $S_n$ est un sous-espace vectoriel de
    $M_n(\R)$. On rappelle que $M_n(\R)$ vient avec une base canonique
    qu'on note $\{E_{ij}\mid 1 \leq i, j \leq n\}$ où $E_{ij}$ est la
    matrice contenant que des coefficients nuls sauf en position
    $(i,j)$ où on trouve $1$. Une matrice $A = (a_{i, j})$ se
    décompose le long de la base précédente comme suit
    \[
    A = \sum_{i, j} a_{ij}E_{ij}.
    \]
    La matrice $A$ est symétrique si $A = A^T$. En utilisant le fait
    $E_{ij}^T = E_{ji}$, on obtient de la décomposition précédente le
    fait que $A$ est symétrique si et seulement si
    \[
    A = \sum_{j \geq i} a_{ij}\left(E_{ij} + E_{ji}\right).
    \]
    Ainsi $\{E_{ij}\mid 1 \leq i, j \leq n, j \geq i\}$ est une
    famille génératrice de $S_n$. Il n'est pas difficile de voir
    qu'elle est également libre, car toutes combinaisons linéaire
    nulle de cette famille est une combinaison linéaire nulle de la
    base de $M_n(\R)$ donnée par les $E_{ij}$. On a donc exhibé une
    base de $S_n$, elle contient $n(n+1)/2$ éléments.

    Vous pouvez, si le coeur vous en dit, montrer que $M_n(\R)$ est la
    somme directe de $S_n$ et du sous-espace vectoriel des matrices
    antisymétriques.
  \item Soit $\theta \in [0, 1]$ et $A, B \in S_n^+$. On veut montrer
    que $(1-\theta)A + \theta B \in S_n^+$, c'est-à-dire que pour tout
    $x \in \R^n$
    \[
    x^T\big((1-\theta)A + \theta B\big)x \geq 0.
    \]
    Or le terme de gauche est donné par
    \[
    (1-\theta)\underbrace{x^TAx}_{\geq 0} + \theta\underbrace{x^TBx}_{\geq 0}
    \]
    d'où le résultat. Le cas de $S_n^{++}$ suit le même procédé.
  \item Oui, il suffit d'adapter le raisonnement précédent.
  \end{enumerate}
\end{solution}

\end{document}

%%% Local Variables:
%%% mode: latex
%%% TeX-master: t
%%% End:
