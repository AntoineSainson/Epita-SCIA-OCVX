\documentclass[11pt, a4paper]{article}

\usepackage[french]{babel}
\usepackage{fancyhdr}
\usepackage[margin=.8in]{geometry}

\usepackage{Style/TeXingStyle}

\pagestyle{fancy}
\renewcommand{\headrulewidth}{1.5pt}
\renewcommand{\footrulewidth}{0.5pt}
\fancyhead[L]{EPITA\_ING2\_2020\_S8}
\fancyhead[R]{Majeures SCIA \& IMAGE}
\fancyhead[C]{OCVX}
\fancyfoot[C]{\thepage}
\fancyfoot[L]{2019}
\fancyfoot[R]{\footnotesize{\textbf{Chargés de cours :} \textsc{B.~DUDIN} \& \textsc{G.~TOCHON}}}

\pretitle{\vspace{-2.5\baselineskip} \begin{center}}
\title{%
  { \huge Un peu de calcul différentiel}%
}
\posttitle{
\end{center}
\rule{\textwidth}{1.5pt}
\vspace{-3\baselineskip}
}
\author{}
\date{}

\pdfinfo{
   /Author (Bashar Dudin)
   /Title  (Calcul Différentiel - 2019)
   /Subject (SCIA/IMAGE - Optimisation convexe)
}

\setlength\parindent{0pt}

\begin{document}

\maketitle\thispagestyle{fancy}

\begin{abstract}
  On souhaite généraliser la démarche adoptée pour résoudre
  géométriquement certains programmes linéaires au cas des problèmes
  d'optimsation non-linéaires. L'objectif est de trouver des
  hyperplans d'appui aux sous-niveaux d'une fonction objectif ; ils
  indiquent une direction de recherche pour minimiser celle-ci. La
  définition de ces hyperplans d'appui s'effectue par une étude locale
  des fonctions objectifs, une étude qui généralise l'apport de la
  dérivée d'une fonction numérique à la compréhension du comportement
  de celle-ci en un point.
\end{abstract}

\tableofcontents

\section{Normes sur $\R^n$}
\label{sec:nomesRn}

Une norme sur un espace vectoriel apporte une manière de mesurer la
\textit{longueur} d'un vecteur tout en respectant un minimum la
structure vectorielle. Elle permet en particulier de définir une
notion de distance entre deux point de $\R^n$ par la \textit{longueur}
du vecteur qui les relie. Pouvoir changer de mesure de longueur
suivant les problèmes qu'on attaque est crucial lorsque l'on s'attaque
à des problèmes d'optimisation. À la fois pour pouvoir modéliser les
problèmes en jeu ; comment mesure la différence entre deux mots? Ou
pour accélérer la convergence de certains algorithmes
d'apprentissages.

\begin{defn}
  Une norme sur $\R^n$ est une application $\|\cdot\| : \R^n \to \R_+$
  telle que:
  \begin{enumerate}
  \item $\|x\| = 0 \Leftrightarrow x = 0$ ;
  \item $\forall \lambda \in \R$, $\forall x \in \R^n$,
    $\|\lambda x\| = |\lambda|\|x\|$ (relation d'\textit{homogéinité});
  \item $\forall x \in \R^n$, $\forall y \in \R^n$,
    $\|x + y\| \leq \|x\| + \|y\|$ (\textit{Inégalité triangulaire}).
  \end{enumerate}
\end{defn}
\begin{question}
  Interpréter chacune des propriétés suivantes avec vos propres mots.
\end{question}
L'inégalité triangulaire donne lieu à une autre inégalité, appelée
\textit{inégalité triangulaire inversée} qui peut parfois être
utile\footnote{Il est très probable qu'elle ne fasse que partie de
  votre culture ...}:
\[
  \forall x, y \in \R^n, \qquad \big| \|x\| -\|y\| \big| \leq \|x -
  y\|.
\]
\begin{question}
  Représenter cette inégalité géométriquement. Essayer de la déduire
  de l'inégalité triangulaire.
\end{question}
Les trois normes les plus fréquentes d'utilisation sont les trois
suivantes, elles sont respectivement qualifiées de normes $1$, $2$ et
infinie.
\begin{enumerate}
\item $\forall x \in \R^n$, $\| x \|_1 = \sum_{i=1}^n |x_i|$ ;
\item $\forall x \in \R^n$,
  $\| x \|_2 = \sqrt{\sum_{i=1}^n x_i^2} = \sqrt{x^Tx}$ ;
\item $\forall x \in \R^n$, $\|x \|_{\infty}$,
  $\max\{|x_i| 1 \leq i \leq n\}$.
\end{enumerate}
Les définitions des normes $1$ et $2$ vont pouvoir se généraliser pour
tout $p \geq 1$\footnote{Il est tout à fait naturel de se poser la
  question de savoir pourquoi $p \geq 1$! :)}, par
\[
  \forall x \in \R^n, \qquad \|x \|_p = \left(\sum |x_i|^p
  \right)^{\frac{1}{p}}.
\]
La norme $\|\cdot\|_p$ est communément appelée la norme $p$. Montrer
le fait que c'est une norme est uniquement délicat pour ce qui est de
l'inégalité triangulaire, la preuve de ce fait se base sur des
inégalités de convexités dites de \textsc{Hölder}, on n'abordera pas
cette preuve dans ce cours.Les normes $p$ ne seront que très
localement utilisées en dehors des cas standards des normes $1$, $2$
et $+\infty$, il est cela dit usuel d'en connaître la définition.

À partir d'une norme on va être en mesure de définir:
\paragraph{Une notion de \textit{distance} entre deux points de
  $\R^n$} par norme du vecteur de qui relie les deux points en
question. Plus formellement étant donné une norme $\|\cdot\|$ on note
\[
  \forall x, y, \qquad d_{\|\cdot\|}(x, y) = \| x - y \|.
\]
Dans le cas des normes $p$ on se contente d'indiquer $p$ en indice.
\begin{question}
  Représenter graphiquement les distances $1$, $2$ et $+\infty$
  entre deux vecteurs de $\R^2$.
\end{question}
\paragraph{Une notion de \textit{voisinage} d'un point de $\R^n$} au
sens de la norme utilisée. Cette notion éminemment liée à la
première, on l'isole ici parce qu'elle apporte un point de vue
auquel vous n'avez pas encore été confrontés. Étant donné un nombre
réel $\varepsilon > 0$ on va qualifier $\varepsilon$-voisinage d'un
point $x \in \R^n$ tous les points à distance (au sens de la norme
utilisée) au plus $\varepsilon$ de $x$. Dans le jargon, on appelle
boule ouverte de rayon $\varepsilon$ et centrée en $x$ cette notion,
pour une norme ambiante $\|\cdot\|$ on note
\[
  B_{\|\cdot\|}(x, \varepsilon) = \{y \in \R^n \mid d_{\|\cdot\|}(x, y) <
  \varepsilon\}
\]
La boule fermée centrée en $x$ et de rayon $\varepsilon$ est la
notion correspondante avec les points à distance $\varepsilon$
incluses,
\[
  \overline{B}_{\|\cdot\|}(x, \varepsilon) = \{y \in \R^n \mid d_{\|\cdot\|}(x,
  y) \leq \varepsilon\}.
\]
Dans le cas des normes $p$ on se contente d'indexer les boules par
$p$.

La notion de $\varepsilon$-voisinage permet d'exprimer des
phénomènes de passage à la limite et d'études locales. Dire qu'une
suite $(u_k)_{k \in \N}$ de points de $\R^n$ converge vers
$\ell \in \R^n$, au sens d'une norme $\|\cdot\|$, correspond au fait
de dire que pour tout $\varepsilon$-voisinage $B(\ell, \varepsilon)$
de $\ell$, il existe un rang $N$ à partir duquel tous les éléments
de la suite $u_k$ sont dans $B(\ell, \varepsilon)$.
\begin{question}
  \begin{itemize}
  \item Dessiner les boules unités
    $\overline{B}_p(\underline{0}, 1)$ pour
    $p \in \{1, 2, \infty\}$.
  \item Montrer que les $\varepsilon$-voisinages d'une norme
    $\|\cdot\|$ sur $\R^n$ sont convexes.
  \item La définition d'un équivalent à une norme $p$ pour $p < 1$
    définit-il une norme?
  \end{itemize}
\end{question}
On reprend plus en détails quelques extensions des notions liées aux
études locales de fonctions en suites dans $\R$.
\begin{exmp}[Convergence d'une suite.]
  Une suite $(u_k)_{k\in \N}$ dans $\R^n$ converge vers un point
  $\ell \in \R^n$, au sens d'une norme $\|\cdot\|$ si
  \[
    \forall \varepsilon > 0, \exists N \in \N, \quad k \geq N
    \Rightarrow \|u_k - \ell \| < \varepsilon.
  \]
  Cette défintion se traduit telle quelle par:
  \[
    \forall \varepsilon > 0, \exists N \in N \quad k \geq N
    \Rightarrow u_k \in B(\ell, \varepsilon).
  \]
  Ou encore: pour tout $\varepsilon > 0$ il existe un rang $N$ à
  partir duquel tous les éléments de la suite $u_k$ sont dans le
  $\varepsilon$-voisinage de $\ell$ pour la norme $\|\cdot\|$.

  Dans le cas de la norme infinie, la définition précédente s'écrit
  explicitement comme
  \[
    \forall \varepsilon > 0, \exists N \in \N, \quad k \geq N
    \Rightarrow \max_{1 \leq i \leq n}\|u_{k, i} - \ell_i \| < \varepsilon.
  \]
  où $u_{k, i}$ (resp. $\ell$) est la composante le long de la
  coordonnée $i$ du vecteur $u_k$ (resp. $\ell_i$). Cette définition
  peut encore être réécrite
  \[
    \forall \varepsilon > 0, \exists N \in \N, \quad k \geq N
    \Rightarrow \forall i \in \{1, \ldots, n\}, \|u_{k, i} - \ell_i \| <
    \varepsilon.
  \]
  Un peu de réflexion permet de voir que cette dernière est
  équivalente
    \[
      \forall i \in \{1, \ldots, n\}, \forall \varepsilon > 0, \exists
      N \in \N, \quad k \geq N \Rightarrow \|u_{k, i} - \ell_i \| <
      \varepsilon.
  \]
  Autrement dit, la suite $(u_k)$ converge vers $\ell$, au sens de la
  norme infinie, si et seulement si, chacune des suites
  \textbf{numériques} $(u_{k, i})$ converge vers $\ell_i$.
\end{exmp}
\begin{exmp}[Limite d'une fonction en un point.]
  On considère deux normes $\|\cdot\|_\alpha$ et $\|\cdot\|_\beta$
  respectivement sur $\R^n$ et $\R^m$. Une fonction
  $f : (\R^n, \|\cdot\|_\alpha) \to (\R^m, \|\cdot\|_\beta)$ a une
  limite $\ell \in \R^m$ en un point $a \in \R^n$ si
  \[
    \forall \varepsilon > 0, \exists \eta > 0, \quad x \neq a
    \mathrm{ et } \|x - a\|_\alpha \eta \Rightarrow \|f(x) -
    \ell\|_\beta < \varepsilon.
  \]
  Chose qu'on peut encore exprimer par
  \[
    \forall \varepsilon > 0, \exists \eta > 0, \quad x \in
    B_{\|\cdot\|_\alpha}(a, \eta)\setminus \{a\} \Rightarrow f(x) \in
    B_{\|\cdot\|_\beta}(\ell, \varepsilon),
  \]
  ou encore pour tout $\varepsilon$-voisinage $B$ de $\ell$ pour la
  norme $\|\cdot\|_\alpha$ il existe un $\eta$-voisinage de $a$, a
  étant exclu dont tout élément a une image dans $B$.
\end{exmp}
\begin{exmp}[Continuité d'une fonction en un point.]
  La continuité d'une fonction
  $f : (\R^n, \|\cdot\|_\alpha) \to (\R^m, \|\cdot\|_\beta)$ en un
  point $a \in \R^n$ s'écrit
  \[
    \forall \varepsilon > 0, \exists \eta > 0, \quad \|x - a\|_\alpha
    \eta \Rightarrow \|f(x) - f(a)\|_\beta < \varepsilon.
  \]
  Autrement dit pour tout $\varepsilon$-voisinage $B$ de $f(a)$ pour
  la norme $\|\cdot\|_\alpha$ il existe un $\eta$-voisinage de $a$
  dont tout élément a une image dans $B$.
\end{exmp}
\begin{question}
  \begin{itemize}
  \item Donner des exemples de fonctions continues (pour les normes de
    votre choix) de $\R^2$ dans $\R$.
  \item Comment peut-on en construire d'autres à partir de celles-ci?
    Que dire de l'addition, la multiplication, le quotient ou la
    composition de fonctions continues?
  \item Comment tester la continuité de fonctions à plusieurs
    variables en se ramenant au cas des fonctions numériques?
  \item Est-ce que \c{c}a marche tous le temps?
  \end{itemize}
\end{question}
\begin{exmp}[Comparaisons en un point.]
  Une fonction
  $f : (\R^n, \|\cdot\|_\alpha) \to (\R^m, \|\cdot\|_\beta)$ est un
  $o$ en un point $a \in \R^n$ d'une fonction $g$, sur les mêmes
  espaces et par rapport aux mêmes normes, s'il existe une fonction
  $\epsilon : \R^n \to \R$ telle que $f = \epsilon g$ avec
  $\lim_{t \to a }\epsilon(t) = 0$. Quand $g$ n'est pas nulle sur un
  voisinage de $a$ la définition précédente est équivalente au fait que
  \[
    \lim_{t \to a} \frac{\|f(t)\|_\alpha}{\|g(t)\|_\beta}  = 0.
  \]
  On écrit dans ce cas que $f$ est un $o_a(g)$. On laisse tomber le
  point $a$ quand celui-ci est clair du contexte.

  Pour obtenir les notions d'équivalence et de $\mc{O}$ en un point
  $a$, il suffit de remplacer les limites de $\epsilon$ ou de quotient
  respectivement par $1$ et une constante positive quelconque.
\end{exmp}
\begin{question}
  Justifier les affirmations suivantes:
  \begin{enumerate}
  \item $\langle h, h \rangle$ est un $o_{\underline{0}}(h)$;
  \item
    $\sin\big(\langle h, h \rangle\big) \sim_{\underline{0}} \langle
    h, h \rangle$.
  \end{enumerate}
\end{question}
À ce stade on est en droit de se demander si le choix des normes avec
lesquels on travaille a une incidence sur les limites des suites qu'on
étudie (une même suite convergerait pour une norme et pas pour une
autre), les fonctions qu'on regarde (une même fonction serait continue
pour une norme et pas pour une autre) etc. Le théorème suivant permet
de se rassurer sur ce point (du moins en dimension finie).
\begin{defn}
  Deux normes $\|\cdot\|_\alpha$ et $\|\cdot\|_\beta$ sur $\R^n$ sont
  dites \emph{équivalentes} s'il existe des constants $c$,
  $C \in \R_+^*$ telles que
  \[
    \forall x \in \R^n, \qquad c\|x\|_\alpha \leq \|x\|_\beta \leq
    C\|x\|_\alpha.
  \]
\end{defn}
En reprenant les exemples précédents, il est relativement simple de se
convaincre que deux normes équivalentes donnent des suites de mêmes
natures, les mêmes fonctions continues et les mêmes comparaisons.
\begin{question}
  Montrer que les normes $1$, $2$ et $\infty$ sont équivalentes.
\end{question}
\begin{thm}
  Toutes les normes sur $\R^n$ sont équivalentes.
\end{thm}
\begin{rem}
  La preuve de ce théorème est en dehors du périmètre de ce
  cours. Elle nécessite certaines notions de topologie qui vous
  manquent ; notamment la notion de compacité. On se contentera du
  résultat.
\end{rem}
Le fait que les normes soient équivalentes en terme de questionnement
topologique (les études des phénomènes locaux) ne signifie pas
qu'utiliser l'une ou l'autre en modélisation revient au même.
\begin{question}
  Quelles normes utiliseriez-vous pour modéliser les problématiques
  suivantes:
  \begin{enumerate}
  \item Le calcul de la distance que doit parcourir un oiseau entre
    deux coordonnées GPS pas trop éloignées.
  \item Le calcul de la distance que parcours un touriste à Manhattan
    entre deux musée.
  \item Le calcul du nombre de modifications nécessaires (lettre à
    lettre) pour changer un mot en un autre.
  \end{enumerate}
\end{question}

\section{Différentiabilité et différentielle en un point}
\label{sec:diffetdiff}

Cette série de questions est calculatoire. Savoir calculer la
différentielle, le gradient (plus généralement la jacobienne) ou la
dérivée directionnelle d'une fonction à plusieurs variables en tout
point (ou du moins là où cela fait sens) est nécessaire.

\begin{question}
  Calculer la différentielle de l'application sur $\R$,
  $f(x) = \frac{\sin(x)}{x^2+1}$ en tout point $x$. Quelle est la
  différence avec la dérivée de $f$.
\end{question}

\begin{question}
  Calculer la différentielle des fonctions suivantes:
  \begin{enumerate}
  \item $f :\R^n \rightarrow \R$ donnée par l'expressoin
    $X \mapsto X^TAX$ ; , où $A$ est une matrice carrée de taille
    $(n, n)$ ;
  \item $g : \R^n \rightarrow \R$ donnée par l'expression
    $X \mapsto 1/(X^TX + 1)$.
  \item $f : \R^n \rightarrow \R$ donnée par l'expressoin
    $X \mapsto \cos^2(X^TAX)$, où $A$ est une matrice carrée de taille
    $(n, n)$ ;
  \item $g : M_n(\R) \rightarrow \R$ donnée par
    $A \mapsto \mathrm{tr}(A)^2$ ;
  \item $h : \mc{M}_n(\R) \rightarrow \mc{M}_n(\R)$ donnée par
    l'expression $B \mapsto \mathrm{tr}(AB)B$ où $A$ est une matrice
    carrée de taille $(n, n)$.
  \end{enumerate}
\end{question}

\begin{question}
  On considère la fonction définie sur $\R^2$ par
  $f(x, y) = \frac{x^2\cos(y)}{y + \sin^2(x) + 1}$. Pour tout
  $\vartheta \in \R$ on désigne par $w_\vartheta$ le vecteur
  $(\vartheta, \vartheta^2)$.
  \begin{enumerate}
  \item Calculer la dérivée directionnelle $\pa_{w_\vartheta}f(0, 0)$
    en fonction de $\vartheta$ ;
  \item Étudier cette fonction de $\vartheta$.
  \item Pouvez-vous interpréter vos résultats géométriquement?
  \end{enumerate}
\end{question}

\begin{question}
  Expliciter le gradient, en tout point ou cela fait sens, des
  expressions différentiables suivantes :
  \begin{enumerate}
  \item $f(x, y) = e^{xy}(x+y)$ ;
  \item $g(x, y, z) = (x + y\ln(z), xyz)$ ;
  \item $h(x, y, z) = \left(\dfrac{x^2\sin(xy)}{z^2 + 2}, \tan(xyz)\right)$ ;
  \item $f(x, y, z) = \dfrac{x + y + z}{x+y^2+1}$ ;
  \item $g(x, y) = \dfrac{\cos(xy)}{\sqrt{x^2 + y^2}}$ ;
  \item $h(x, y, z) = \exp(xy - z)$.
  \end{enumerate}
\end{question}

\begin{question}
  Soit $f : \R^3 \to \R$ une fonction différentiable en tout point de
  $\R^3$. On considère la fonction $g : \R^3 \to \R$ donnée pour tout
  $(x, y, z) \in \R^3$ par
  \[
  g(x, y, z) = f(x-y, y-z, z-x).
  \]
  La fonction $g$ est différentiable en tout point car composée de
  fonction différentiables. Montrer qu'on a la relation
  \[
  \frac{\pa g}{\pa x}(a,b, c) + \frac{\pa g}{\pa y}(a, b, c) + \frac{\pa
    g}{\pa z}(a, b, c) = 0
  \]
  pour tout point $(a, b, c) \in \R^3$.
\end{question}

\subsection{Interprétations géométriques}
\label{sec:interprétationsgeoms}

\begin{question}
  Trouver les points sur le paraboloïde $z = 4x^2 + y^2$ où le plan
  tangent est parallèle au plan $x + 2y + z = 6$. Faire de même avec le plan
  $3x + 5y -2z = 5$.
\end{question}

\begin{question}
  Un étudiant malheureux trouve pour plan tangent à la surface donnée
  par $z = x^4-y^2$ au point $(2, 3, 7)$ la réponse
  \[
  z = 4x^3(x-2) - 2y(y-3) + 7.
  \]
  \begin{enumerate}
  \item Sans calcul, pourquoi est-ce faux?
  \item Donner la réponse correcte.
  \item D'où venait la confusion de l'étudiant?
  \end{enumerate}
\end{question}

\subsection{Optimiser, toujours}

On revient sur les problèmes d'optimisation qu'on est désormais en
mesure d'optimiser à l'aide des outils de calcul différentiel abordés.
\begin{question}
On considère la fonction différentiable $f : \R^2 \to \R$ donnée par
\[
f(x, y) = 3x^2 + y^2.
\]
\begin{enumerate}
\item Représenter les courbes de niveaux $2$ et $4$ de $f$ dans le
  plan euclidien.
\item À quel lieu correspond la condition $f(x, y) \leq 4$ ?
\item On s'intéresse au problème d'optimisation (\emph{P4})
  \[
  \begin{PbOptim}{
      minimiser
    }{
      $2x + y$
    }{
      $3x^2 + y^2 \leq 4$
    }
  \end{PbOptim}
  \]
  Représenter la courbe de niveau de la fonction objectif qui
  correspond à la valeur optimale de (\emph{P4}).
\item Comment trouver le point optimale correspondant à (\emph{P4})?
  Faire le calcul.
\end{enumerate}
\end{question}

\begin{question}
  On considère le problème d'optimisation (\emph{P5}) suivant
  \[
    \begin{PbOptim}{
        minimiser
      }{
        $x + y$
      }{
        $\begin{matrix}
          x+2y \leq 3 \\
          x \in B
        \end{matrix}$
      }
    \end{PbOptim}
  \]
  où $B$ est la partie de $\R^2$ intersection de l'épigraphe de la
  fonction $x \mapsto -\sqrt{x}$ sur $\R_+$ et de la partie
  \[
    \left\{(x, y) \mid y \leq \sqrt{x}\right\}.
  \]
  \begin{enumerate}
  \item Dessiner le lieu admissible du problème de (\emph{P5}).
  \item Représenter la courbe de niveau de la fonction objective qui
    réalise le minimum de (\emph{P5}).
  \item Calculer le point optimal ainsi que la valeur optimale de
    (\emph{P5}).
  \end{enumerate}
\end{question}

\section{Lagrangien et conditions KKT}
\label{sec:lagrangien}

On applique dans la suite la démarche KKT pour résoudre des problèmes
d'optimisation plus délicat à résoudre géométriquement.

\subsection{Des problèmes simples}

\begin{question}
  Résoudre le problème d'optimisation
  \[
  \begin{linearProg}{
      minimiser
    }{
      $x^2 -14x + y^2 - 6y - 7$
    }{
      \systeme{
        x + y \leq 2,
        x + 2y \leq 3
      }
    }
  \end{linearProg}
  \]
\end{question}

\begin{question}
  Une ressource disponible en quantité d doit être affectée à trois
  activités en quantités $x_1$ , $x_2$ et $x_3$
  respectivement. L’allocation de $x_k$ unités de ressource à
  l’activité $k$ procure une recette évaluée par
  $f_k(x_k) = 8x_k - k{x_k}^2$. Calculer l'allocation qui procure la
  plus grande recette dans les deux éventualités suivantes:
  \begin{enumerate}
  \item la quantité $d$ est complètement utilisée ;
  \item l'excédant quand $d$ n'est pas épuisé est revendu au prix $p$.
  \end{enumerate}
  A-t-on intérêt à utiliser toutes les ressources disponibles?
\end{question}

\subsection{Un problème débile}

Une entreprise fabrique deux modèles de clés USB, les modèles $X$ et
$Y$. Le modèle $X$ se vend à $1$\euro{} pièce, le modèle $Y$ se vend pour
sa part à $3$\euro{} pièce. Le coût de fabrication en \euro{} est donné par
l'expression :
\[
C(x, y) = 5x^2 + 5y^2 - 2xy - 2x - 1000.
\]
où $x$ est le nombre de clés USB de type $X$ et $y$ celui de type
$Y$. On suppose que toutes les clés USB ont été toutes écoulées sur le
marché.

\begin{question}{6}
  \begin{enumerate}
  \item Soit $(x, y) \in (\R_+^*)^2$. Déterminer le profit $P(x, y)$
    réalisé par l’entreprise lorsqu’elle a vendu $x$ clés de modèle $X$
    et $y$ clés de modèle $Y$.
  \item Étudier la convexité de la fonction de profit $P$ sur
    $(\R_+^*)^2$. Vous pouvez vous aider de la hessienne de $P$ en
    tout point.
  \item La capacité de production de l’entreprise est au total de $20$
    clés par jour\footnote{C'est des clés fait main!}. Décrire le
    problème de maximisation du profit de l'entreprise comme un
    problème d'optimisation convexe.
  \item En supposant que l’entreprise tourne à plein régime, trouver
    la répartition optimale entre les modèles de type $X$ et $Y$
    permettant de maximiser le profit quotidien. Calculer dans ce cas
    le profit réalisé. \textit{Vous êtes invités, en un premier temps,
      à ne pas tenir compte des contraintes de positivité de $x$ et
      $y$.}
  \item Pourquoi peut-on se dispenser des contraintes de positivité ?
  \end{enumerate}
\end{question}

\subsection{Un problème géométrique}

\begin{question}
  \begin{enumerate}
  \item
    Déterminer le rectangle de plus grande surface inscrit dans
    l'ellipse d'équation
    \[
    \frac{x^2}{a} + \frac{y^2}{b} = 1.
    \]
  \item
    Faire de même avec le parallélépipède de plus grand volume inscrit
    dans l'ellipsoïde d'équation
    \[
    \frac{x^2}{a} + \frac{y^2}{b} + \frac{z^2}{c} = 1.
    \]
  \end{enumerate}
\end{question}

\section{Hessienne d'une fonction}
\label{sec:hessienne}

Dans cette section, on revient sur un point laissé de côté lors du
cours: vous avez su lors de vos études secondaires étudier les types
de point critiques que vous obteniez par l'annulation de la dérivée
d'une fonction numérique à l'aide de la dérivée seconde. Cette
démarche s'étend au cas des dimensions supérieures. Pour cela il nous
faut généraliser la notion de dérivée seconde.
\begin{defn}
  Soit $f : \R^n \to \R$ une fonction différentiable en tout point de
  son domaine de différentielle continue. On dit que $f$ est $2$-fois
  différentiable en un point $a$ du domaine de $f$ si le $\nabla f$
  est différentiable. Dans cas on appelle \emph{hessienne} de $f$ en
  $a$ la jacobienne de $\nabla f$. Elle est notée $\nabla^2 f(a)$.
\end{defn}
On regroupe les notations des coefficients de la matrice hessienne ;
on écrira
\[
  \frac{\partial^2 f}{\partial x^2} = \frac{\partial}{\partial
    x}\left(\frac{\partial f}{\partial x}\right)
\]
et
\[
  \frac{\partial^2 f}{\partial x\partial y} = \frac{\partial}{\partial
    x}\left(\frac{\partial f}{\partial y}\right).
\]
Quand elle existe et \textit{quand ses composantes sont continues}, la
hessienne d'une fonction $f$ en un point $(a, b)$ définit une forme
quadratique\footnote{Théorème de Schwarz} qu'on appelle
\emph{différentielle seconde} de $f$ en $(a, b)$. Elle donne une
approximation du second ordre de $f$ au voisinage de $(a, b)$.  La
hessienne permet une étude plus fine des points critiques de la
fonction $f$ ainsi qu'une caractérisation pratique des conditions de
convexité de celle-ci.
\begin{prop}
  Soit $f : U \to \R^m$ une application $2$ fois différentiable en un
  point $a$ de l'ouvert $U$. Alors pour $h$ assez petit
  \begin{equation}
    \label{eq:DL2}
    f(a+h) = f(a) + \nabla f(a) h + \dfrac{1}{2} h^T\nabla^2f(a)h  + o_{\underline{0}}(\|h\|^2).
  \end{equation}.
\end{prop}
\begin{question}
  Soient $f : \R \to \R$ une fonction $2$-fois dérivable et $P$ une
  matrice symétrique.  Quelle est la différentielle seconde, en tout
  point, de l'application $f_P$ définie sur $\R^n$ par
  $x \mapsto f(x^TPx)$ ?
\end{question}
\begin{question}
  \label{eq:hessienne}
  Calculer les hessiennes en tout point des fonctions à valeurs
  réelles suivantes:
  \begin{enumerate}
  \item $f(x, y) = x^2 + xy + y^2 + \displaystyle{\frac{x^3}{4}}$;
  \item $f(x, y) = x^3 + y^3$;
  \item $f(x, y) = x^2y - \displaystyle{\frac{x^2}{2}} - y^2$;
  \item $f(x, y, z) = x^2 + 3y^2 - z^2 + 2xy - 5yz$;
  \item $f(x, y, z) = \ln\big(e^x + e^y +e^z\big)$.
  \end{enumerate}
\end{question}

\subsection{Convexité}

Soit $f : U \to \R$ une fonction définie et différentiable sur un ouvert convexe
$U \subset \R^n$. Alors $f$ est convexe si et seulement si
\begin{equation}
  \label{eq:convexeOrdUn}
  \forall x, y \in U, \quad f(x) - f(y) \geq \nabla{f}(x)(x-y).
\end{equation}
On rappelle que ce critère \eqref{eq:convexeOrdUn} a une
interprétation géométrique simple: l'hyperplan tangent au graphe de
$f$ en un point $(x, f(x))$ est un hyperplan d'appui. En effet, cette
dernière condition se traduit en un point $x \in U$ par
\[
(\nabla{f}(x), -1)  \begin{pmatrix}  y - x \\ f(y) - f(x) \end{pmatrix} \leq 0
\]
qui donne, une fois la multiplication matricielle explicitée, la
relation \eqref{eq:convexeOrdUn}. On rappelle que le vecteur
$(\nabla{f}(x), -1)$ est un vecteur normal définissant l'hyperplan
tangent au graphe de $f$ au point $(x, f(x))$. Notez au passage le fait
que $\nabla{f}(x)$, $x$ et $y$ sont des vecteurs dans $\R^n$ ;
l'écriture ci-dessus est donc une écriture par blocs.

Ce critère d'ordre $1$ est pratique mais pas toujours facile à
manier. Quand $f$ est une fonction $2$ fois différentiable, tout comme
dans le cas d'une fonction de $\R$ dans $\R$, la positivité de la
différentielle seconde nous apporte une information sur la convexité
de $f$. Plus précisément:
\begin{prop}
  Soit $f : U \to \R$ une fonction $2$ fois différentiable définie sur
  un ouvert convexe $U \subset \R^n$. Alors $f$ est convexe si et
  seulement si
  \begin{equation}
    \label{eq:convexeOrdDeux}
    \forall x \in U, \quad \nabla^2{f}(x) \geq 0.
  \end{equation}
\end{prop}

Ainsi, un point critique d'une fonction convexe ne peut être qu'un
minimum local, donc global. Les autres cas de figures sont ceux d'un
maximum local (si la hessienne est négative) ou d'un \emph{point
  selle} quand la hessienne n'est négative ni positive.

\begin{question}
  Les fonction de la question \eqref{eq:hessienne} sont-elles convexes ?
  Justifier.
\end{question}



\end{document}
%%% Local Variables:
%%% mode: latex
%%% TeX-master: t
%%% End:
